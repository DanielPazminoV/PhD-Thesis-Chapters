
\chapter{Fire weather in the Ecuadorian Andean region}
\newpage{}


\section{Introduction}

Chapters 5 and 6 investigated fire weather variability in Victoria,
Australia. This research used the McArthur Forest Fire Danger Index
(FFDI) to represent fire weather. The chapters also describe the development
of the Victorian Seasonal Bushfire Index (VSBI). The VSBI is a forecasting
tool and alternative fire weather metric. The last two chapters also
examined climate-bushfire relationships in Victoria.

This chapter investigates fire weather in Ecuador. The analyses use
some of the methods applied for Victoria. The investigation examines
if the FFDI can accurately represent fire weather in Ecuador. The
study of fire weather changes in this region used this index. Additionally,
this research explores the relationship between ENSO and fire weather.
Finally, it assessed if the 20CR and ERA-20C reanalyses capture the
observed fire weather variability. 


\section{Data and methodology}


\subsection{Data}


\subsubsection{Weather station}

This study required weather station (WS) data from the Ecuadorian
Andean region. Figure \ref{fig:Location of weather stations in the Ecuadorian andean region}
illustrates the spatial distribution of the selected weather stations
(see the Appendix B for further details). Complete daily records were
available only for six stations for the period 1997-2012. Each of
the stations provided data on an eight-hourly basis. Therefore, the
hour selected for the computations was 1 p.m. local time (GMT 18:00).
The analyses required the use of the four basic fire weather variables:
temperature, relative humidity, wind speed and precipitation. 

\begin{figure}[h]
\noindent \begin{centering}
\includegraphics[scale=0.4]{Chapter_7/Figures/Maps/Weather_Stations_Ecuador.png}
\par\end{centering}

\caption[Location of weather stations in the Ecuadorian andean region]{Location of weather stations in the Ecuadorian andean region \label{fig:Location of weather stations in the Ecuadorian andean region}}
\end{figure}



\subsubsection{Reanalysis}

This investigation used the two reanalysis products analysed in previous
chapters. These datasets are version two of the Twentieth Century
Reanalysis Project (20CR) \citep{Compo2011} and the ERA-20C reanalysis
from the European Reanalysis of Global Climate Observations ERA-Clim
project \citep*{Stickler2014,Stickler2014a}. These reanalysis products
yield climate fields with a spatial resolution of 2\textsuperscript{o}x2\textsuperscript{o}and
1.5\textsuperscript{o}x1.5\textsuperscript{o} respectively. Chapters
5 and 6 provide further details about these datasets. 


\subsection{Methodology}


\subsubsection{Reliability assessment of the FFDI}

The study examines the skill of the FFDI to represent fire weather
in Ecuador. This index was selected because its calibration used the
\textit{Eucalyptus} vegetation. These species dominate the landscape
of bushfire-prone areas in Ecuador (see Chapter 2). The validations
comprised the period with available weather station data (1997-2012).
Additionally, the analysis used the cumulative FFDI metric (FFDI\textsubscript{cum}).
Preliminary tests showed that including all days in the fire season
in the analyses allows to better visualize fire weather changes in
this region. Unfortunately, the lack of fire activty records in Ecuador
does not allow a direct verification of the FFDI's accuracy. 

Testing the skill of the FFDI to represent fire weather in the Ecuadorian
tropical Andes required several analyses. First of all, the study
verified if linear correlations show physical consistency among the
relationships between fire weather variables and the index oucomes.
The daily values of each fire weather variable were spatially averaged
to perform the FFDI computations and correlations. The subsequent
analyses used these averaged data. 

The investigation examined if the FFDI captures the seasonal variability
of fire weather in the Ecuadorian Andes. This analysis required the
computation seasonal FFDI time series. This analysis aimed to verify
if the index agrees with local reports that suggest higher fire activity
during the season July-August-September (see Chapter 2). Morevover,
the background chapter also described that 2012 was a particularly
dangerous season. Thus, the FFDI time series should also exhibit ``high''
fire danger during this year. 

Additionally, this research explored the monthly variability of fire
weather. This evaluation aimed to complement the seasonal time series
analysis. If the FFDI is an accurate metric it should display high
fire danger during July, August and September. Additionally, this
investigation tested methods described in previous chapters to attempt
extending the analysis further back in time. 


\subsubsection{Fire weather and the El Ni\~no-Southern Oscillation}

This investigation also explored the relationship between fire weather
and the El Ni\~no-Southern Oscillation. The study compares the FFDI\textsubscript{cum}
results (using observations) with annual ENSO categories \citep{BoM-ENSO-Years-2016}.
This step attempts to answer during which ENSO phase fire danger is
higher in this region. 


\subsubsection{Fire weather analysis using reanalysis data}

The study aimed to extend the analysis before the observation period.
The first step was to compare the reanalysis data against observations.
Plotting probability density functions (PDFS) and seasonal time series
facilitated the analysis. The PDFs allowed the examination of the
distribution of data and detect deviations from the mean. On the other
hand, time series facilitated comparing trends, magnitudes and correlate
data. The variables that showed significant deviations from the observed
climatology were bias-corrected. This process used linear scaling
techniques. The analyses compared the FFDI computations using observations
and bias-corrected reanalysis data. As in the previous step, plotting
probability density functions and time series allowed assessment of
the FFDI results. 


\section{Results}

Figure \ref{fig:Pearson correlation coefficients between the Forest Fire Danger Index and its variables for July-August-September in the Ecuadorian andean region during the period 1997-2012.}
shows linear correlation coefficients between the FFDI and its variables.
The analysis used weather station data during the season July-August-September
(JAS) for the period 1997-2012. The graph displays coloured circles
for correlations with statistical significance above the 5\% level.
The results demonstrate physical consistency in the relationships
between meteorological variables and fire weather represented by the
FFDI. Additionally, the index shows that temperature and relative
humidity are the most sensitive parameters in this region. These variables
have correlation coefficients with the FFDI of 0.87 and -0.83 respectively.
On the other hand, precipitation (r=-0.79) and wind speed (r=0.69)
are less sensitive. 

\begin{figure}[h]
\noindent \begin{centering}
\includegraphics[scale=0.7]{Chapter_7/Analysis/FFDI_variables/FFDI_variables_correlations}
\par\end{centering}

\caption[Pearson correlation coefficients between the Forest Fire Danger Index
and its variables for July-August-September in the Ecuadorian Andean
region during the period 1997-2012]{Pearson correlation coefficients between the Forest Fire Danger Index
and its variables for July-August-September in the Ecuadorian Andean
region during the period 1997-2012. Coloured circles represent values
with statistical significance at the 5\% level. \label{fig:Pearson correlation coefficients between the Forest Fire Danger Index and its variables for July-August-September in the Ecuadorian andean region during the period 1997-2012.}}


\end{figure}


Figure \ref{fig:Cumulative Forest Fire Danger Index values (FFDI) time series in the Ecuadorian andean region for the period 1997-2012}
shows seasonal FFDI time series computed on the Ecuadorian highlands.
The calculation uses averaged data from weather stations in this region.
The graph shows the FFDI\textsubscript{cum} results for four seasons:
July-August-September (JAS), October-November-December (OND), January-February-March
(JFM), and April-May-June (AMJ). As expected, JAS demonstrates greater
fire danger than most seasons. Surprisingly, the season OND shows
a similar fire danger magnitude than JAS. In fact, the two seasons
virtually have equal FFDI\textsubscript{cum} means (158 and 157 respectively).
Additionally, the figure shows that this region experienced the highest
fire danger during the years 2000 and 2008. 

\begin{figure}[h]
\noindent \begin{centering}
\includegraphics[scale=0.4]{Chapter_7/Figures/TS/FFDI_obs_seasons_ec.png}
\par\end{centering}

\caption[Cumulative Forest Fire Danger Index values (FFDI) time series in the
Ecuadorian Andean region for the period 1997-2012]{Cumulative Forest Fire Danger Index values (FFDI) time series in the
Ecuadorian Andean region for the period 1997-2012. The lines show
the results for July-August-September (JAS), October-November-December
(OND), January-February-March (JFM), and April-May-June (AMJ). Red
lines represent the periods with the highest fire danger. \label{fig:Cumulative Forest Fire Danger Index values (FFDI) time series in the Ecuadorian andean region for the period 1997-2012}}
\end{figure}


Figure \ref{fig:Monthly variability of cumulative Forest Fire Danger Index (FFDI) values in the Ecuadorian andean region during the period 1997-2012}
illustrates the monthly variability of fire weather during the period
1997-2012. The figure shows that fire danger starts increasing in
July. In the later months, it continues a steep increase reaching
peaks in September and October. From November to April, fire danger
experiences a consistent decrease. During May and June it plateaus
before the start of the fire season. 

\begin{figure}[h]
\noindent \begin{centering}
\includegraphics[scale=0.5]{Chapter_7/Figures/Monthly_FFDI}
\par\end{centering}

\caption[Monthly variability of cumulative Forest Fire Danger Index (FFDI)
values in the Ecuadorian Andean region during the period 1997-2012]{Monthly variability of cumulative Forest Fire Danger Index (FFDI)
values in the Ecuadorian Andean region during the period 1997-2012.
The graph shows that the highest fire danger months comprises the
period July-November \label{fig:Monthly variability of cumulative Forest Fire Danger Index (FFDI) values in the Ecuadorian andean region during the period 1997-2012} }
\end{figure}


Figure \ref{fig:Cumulative Forest Fire Danger Index (FFDI) during the season July-August-September-October-Novermber (JASON) and its relationship with El Ni=0000F1o-Southern Oscillation in the Ecuadorian Andean region}
shows cumulative FFDI values during the extended season July-August-September-October-November-December
(JASON). Considering the result displayed in Figure \ref{fig:Monthly variability of cumulative Forest Fire Danger Index (FFDI) values in the Ecuadorian andean region during the period 1997-2012},
the analysis included October and November. These two months also
exhibit high FFDI values. Figure \ref{fig:Cumulative Forest Fire Danger Index (FFDI) during the season July-August-September-October-Novermber (JASON) and its relationship with El Ni=0000F1o-Southern Oscillation in the Ecuadorian Andean region}
contrast the results for \textquotedblleft El Ni\~no\textquotedblright{}
and \textquotedblleft La Ni\~na\textquotedblright{} years during the
period 1997-2010. The graph shows that fire danger increases during
\textquotedblleft El Ni\~no\textquotedblright{} years.

\begin{figure}[h]
\noindent \begin{centering}
\includegraphics[scale=0.55]{Chapter_7/Figures/FFDI_ENSO}
\par\end{centering}

\caption[Cumulative Forest Fire Danger Index (FFDI) during the season July-August-September-October-November
(JASON) and its relationship with El Ni\~no-Southern Oscillation in
the Ecuadorian Andean region]{Cumulative Forest Fire Danger Index (FFDI) during the season July-August-September-October-November
(JASON) and its relationship with El Ni\~no-Southern Oscillation in
the Ecuadorian Andean region. Fire danger in this region increases
during ``El Ni\~no'' years. \label{fig:Cumulative Forest Fire Danger Index (FFDI) during the season July-August-September-October-Novermber (JASON) and its relationship with El Ni=0000F1o-Southern Oscillation in the Ecuadorian Andean region} }


\end{figure}


Figure \ref{fig:Probability density functions of fire weather variables in the Ecuadorian andean region during the period 1997-2010}
displays probability density functions of fire weather variables.
The figure compares weather station with original reanalysis data.
The figure also shows the deviations from the mean for each variable
using the 20CR and ERA-20C reanalyses. Both reanalysis datasets display
large deviations. 

\begin{figure}[h]
\noindent \begin{centering}
\includegraphics[scale=0.55]{Chapter_7/Figures/PDFs/PDFs_variables_original}
\par\end{centering}

\caption[Probability density functions of fire weather variables in the Ecuadorian
Andean region during the period 1997-2010]{Probability density functions of fire weather variables in the Ecuadorian
Andean region during the period 1997-2010. The plots use original
reanalysis data. Data source: Twentieth Century Reanalysis (20CR),
European Reanalysis of Global Climate Observations (ERA-20C) and weather
stations in the Ecuadorian andean region (WS). \label{fig:Probability density functions of fire weather variables in the Ecuadorian andean region during the period 1997-2010}}
\end{figure}


Figure \ref{fig:Time series of fire weather variables during the season July-August-September in the Ecuadorian andean region for the period 1997-2010}
shows time series of fire weather variables. The plot includes the
correlation coefficients between time series that use observations
and reanalysis data. These correlations indicate that reanalysis datasets
do not capture climate variability in this region. Overall, the 20CR
data performs better representing the inter-annual variability than
the ERA-20C dataset. The 20CR and WS time series present positive
correlations. Although, these correlations are weak and not statistically
significant. On the other hand, the ERA-20C data shows negative correlations.

\begin{figure}[h]
\noindent \begin{centering}
\includegraphics[scale=0.55]{Chapter_7/Figures/TS/TS_variables_original}
\par\end{centering}

\caption[Time series of fire weather variables during the season July-August-September
(JAS) in the Ecuadorian Andean region for the period 1997-2010]{Time series of fire weather variables during the season July-August-September
(JAS) in the Ecuadorian Andean region for the period 1997-2010. The
plots use original reanalysis data. Data source: Twentieth Century
Reanalysis (20CR), European Reanalysis of Global Climate Observations
(ERA-20C) and weather stations in the Ecuadorian andean region (WS).
\label{fig:Time series of fire weather variables during the season July-August-September in the Ecuadorian andean region for the period 1997-2010}}
\end{figure}


Figures \ref{fig:Bias-corrected probability density functions of fire weather variables in the Ecuadorian andean region during the period 1997-2010}
and \ref{fig:Bias-corrected time series of fire weather variables in the Ecuadorian andean region during the period 1997-2010}
display PDFs and time series\textemdash respectively\textemdash of
fire weather variables using bias-corrected reanalysis data. The bias
correction was applied to those variables that showed the best potential
to be corrected using linear scaling techniques. The figures show
the bias-corection results for maximum temperature (20CR and ERA-20C)
and wind speed (ERA-20C). The application of additive linear scaling
centred the means of these variables. For wind speed (ERA-20C), multiplicative
linear scaling adjusted the data distribution.

\begin{figure}[h]
\noindent \begin{centering}
\includegraphics[scale=0.55]{Chapter_7/Figures/PDFs/PDFs_variables_bc}
\par\end{centering}

\caption[Bias-corrected probability density functions of fire weather variables
in the Ecuadorian Andean region during the period 1997-2010]{Bias-corrected probability density functions of fire weather variables
in the Ecuadorian Andean region during the period 1997-2010. The plots
use bias-corrected (bc) reanalysis data. Data source: Twentieth Century
Reanalysis (20CR), European Reanalysis of Global Climate Observations
(ERA-20C) and weather stations in the Ecuadorian andean region (WS).\label{fig:Bias-corrected probability density functions of fire weather variables in the Ecuadorian andean region during the period 1997-2010}}
\end{figure}


\begin{figure}[h]
\noindent \begin{centering}
\includegraphics[scale=0.55]{Chapter_7/Figures/TS/TS_variables_bc}
\par\end{centering}

\caption[Bias-corrected time series of fire weather variables in the Ecuadorian
Andean region during the period 1997-2010]{Bias-corrected time series of fire weather variables in the Ecuadorian
Andean region during the period 1997-2010. The plots use bias-corrected
(bc) reanalysis data during the season July-August-September (JAS).
Data source: Twentieth Century Reanalysis (20CR), European Reanalysis
of Global Climate Observations (ERA-20C) and weather stations in the
Ecuadorian andean region (WS). \label{fig:Bias-corrected time series of fire weather variables in the Ecuadorian andean region during the period 1997-2010}}
\end{figure}


Figure \ref{fig:Probability density functions of Forest Fire Danger Index values in the Ecuadorian andean region during the period 1997-2010}
shows FFDI probability density functions. The plot compares the results
using observations and reanalysis (original and bias-corrected) data.
The bias-correction improves the distribution of the FFDI computed
with reanalysis datasets. On the other hand, Figure \ref{fig:Time series of cumulative Forest Fire Danger Index values in the Ecuadorian andean region during the period 1997-2010}
displays time series of cumulative FFDI. The plot contrasts FFDI computed
with observations and reanalysis (original and bias-corrected) data.
The results show that both reanalysis fail to capture the observed
fire weather variability. Therefore, the analyses were not extended
further back in time. 

\begin{figure}[h]
\noindent \begin{centering}
\includegraphics[scale=0.55]{Chapter_7/Figures/PDFs/PDFs_FFDI}
\par\end{centering}

\caption[Probability density functions of Forest Fire Danger Index values (FFDI)
in the Ecuadorian Andean region during the period 1997-2010]{Probability density functions of Forest Fire Danger Index values (FFDI)
in the Ecuadorian Andean region during the period 1997-2010. The plots
use original and bias-corrected (bc) reanalysis data. Data source:
Twentieth Century Reanalysis (20CR), European Reanalysis of Global
Climate Observations (ERA-20C) and weather stations in the Ecuadorian
andean region (WS). \label{fig:Probability density functions of Forest Fire Danger Index values in the Ecuadorian andean region during the period 1997-2010}}
\end{figure}


\begin{figure}[h]
\noindent \begin{centering}
\includegraphics[scale=0.55]{Chapter_7/Figures/TS/TS_FFDI}
\par\end{centering}

\caption[Time series of cumulative Forest Fire Danger Index values (FFDI) in
the Ecuadorian Andean region during the period 1997-2010]{Time series of cumulative Forest Fire Danger Index values (FFDI) in
the Ecuadorian Andean region during the period 1997-2010. The plots
use original and bias-corrected (bc) reanalysis data during the season
July-August-September (JAS). Data source: Twentieth Century Reanalysis
(20CR), European Reanalysis of Global Climate Observations (ERA-20C)
and weather stations in the Ecuadorian andean region (WS). \label{fig:Time series of cumulative Forest Fire Danger Index values in the Ecuadorian andean region during the period 1997-2010}}
\end{figure}



\section{Discussion}

The FFDI is a useful metric of fire danger in the Ecuadorian Andes.
The index represented as ``high'' fire danger months those considered
to be of greater fire activity (JAS). It also showed that October
and November can also be dangerous. Additionally, the FFDI shows that
``El Ni\~no'' events are linked to higher fire danger in this region.

The FFDI results showed that the Ecuadorian Andes experience higher
fire danger from July to November. \citet{Estacio2012} reports that
JAS is the typical fire occurrence season in the city of Quito. This
is the capital of the country and where more information about bushfires
is available. Quito is located in the northern Ecuadorian Andes. Therefore,
the fire reporting could be biased towards this region. In fact, there
is scarce information about fire occurrences in other regions of the
Ecuadorian Andes which might occur in October and November. In contrast,
other reports argue that during these months fire activity does occur
(with less intensity) \citep{SecretariadeAmbiente2013}. 

The results using the FFDI do not show that the year 2012 experienced
high fire danger in comparison to other years. This result contrasts
with the local reports which claim that this year experienced severe
fire activity \citep{MinisteriodelAmbiente2013,SecretariadeAmbiente2013}.
Therefore, the intense fire activity in 2012 could be better explained
by factors beyond extreme weather (e.g. higher number of ignitions).
Additionally, more robust tests of the realibility of the FFDI in
this region also require fire activity records. Thus, generating this
informtion is of paramount importance to continue investigating this
problem. 

On the other hand, \textquotedblleft El Ni\~no\textquotedblright{} events
bring higher fire danger to the Ecuadorian Andes. This result is consistent
with the evidence discussed in the literature review. \citet{Garreaud2009}
argues that this region experiences less than average precipitations
during \textquotedblleft El Ni\~no\textquotedblright . In contrast,
temperature is higher than average during these events. \citet{Vuille2000}
and \citet{Francou2004} results agree with this pattern. Additionally,
\citet{Francou2004} showed that \textquotedblleft El Ni\~no\textquotedblright{}
produce negative mass balances on the tropical Andean mountain glaciers. 

The hot and dry conditions that \textquotedblleft El Ni\~no\textquotedblright{}
events produce in the Ecuadorian Andes contrast with its effect on
the coast. \citet{Garreaud2009} argues that a distortion of the Hadley
cell circulation produces this phenomenon. Although, it is intriguing
that the strong \textquotedblleft El Ni\~no\textquotedblright{} of 1997-1998
is not the highest fire danger year in the analysed period (1997-2012).
Extending the analysis period could provide further insights into
this relationship. 

The discussion also aims to further understand the limitations of
20CR and ERA-20C datasets to extend the fire weather analyses. The
caveats of these datasets were acknowledged in previous chapters.
However, the results showed biases that were unexpectedly large. Even
more importantly, these datsets did not show to be capable of capturing
climate variations in time. Therefore, the discussion will attempt
to explain these discrepancies. Topography and the amount of data
assimilated in the reanalysis products are the main aspects examined.
The discussion will also explore the feasibility of using alternative
datasets and future work. 

Topography is likely to cause major biases in the reanalysis data.
Mountain ranges can block circulation and influence regional climate
greatly \citep{Beniston2005}. In fact, mountains can create a climate
of their own \citep{Ekhart1948}. The Ecuadorian Andes comprise the
Eastern and Western ``Cordilleras'' (see Chapter 2). Meteorological
stations in the Ecuadorian Andes are located in the inter-Andean valleys.
The two ``Cordilleras'' surround these valleys. These plains experience
a broad range of microclimates produced by the steep topography \citep{Pourrut1995}.
The results show that 20CR and ERA-20C do not grasp the regional climate
variability caused by this complex geographic region. In contrast,
reanalysis performs well in Victoria, which is a region predominantly
flat. 

Nevertheless, reanalysis datasets have demonstrated better results
in other mountain regions in the world. \citet{Widmann1999} validated
precipitation trends in the north-west United States from the National
Centers for Environmental Prediction (NCEP) reanalysis. \citet{Frauenfeld2005}
showed that the European Centre for Medium-Range Weather Forecasts
(ECMWF) reanalysis (ERA-40) was able to reproduce 2m air temperature
variability in the Tibetan Plateau. Even though ERA-40 underestimated
temperature by 7 \textsuperscript{o}C this dataset demonstrated high
temporal correlation with observations. Therefore, bias-corrections
were feasible. 

These studies showed that reanalysis can capture climate variability
in other high elevation regions. Thus, it is worth discussing the
input data that generated the reanalysis datasets used in this research.
The 20CR assimilates surface pressure and uses sea surface temperature
and sea ice concentrations as boundary conditions \citep{Compo2011}.
The International Surface Pressure Databank version 2 (ISPDv2) generated
the pressure data used on the 20CR \citep*{Cram2015}. On the other
hand, the main data for the development of the ERA-20C dataset was
historical upper-air data \citep*{Stickler2014,Stickler2014a}. Although,
the ERA-20C reanalysis also used data form the ISPDv2 \citep*{Poli2013,Cram2015}. 

South America and Australasia are the populated continents with less
land observation stations for pressure data \citep*{Cram2015}. Furtheremore,
the authors show that only 7 out of 65 institutions that provided
data to the ISPDv2 belong to the Southern Hemisphere. However, this
thesis demonstrated that 20CR and ERA-20C can represent climate variability
in Victoria, Australia. In spite of the overall lack of data in the
Southern Hemisphere, \citet*{Cram2015} showed that Victoria has a
high density of locations for pressure data collection. In contrast,
the Ecuadorian Andean region does not register stations generating
pressure information in the ISPDv2. This database shows that pressure
data from Ecuador is produced only in a couple of stations on the
coast. In fact, most data available from the neighbouring countries
comes from the coast as well. 

On the other hand, ERA-CLIM's data rescue activities do not show progress
in the tropical Andean region. \citet*{Stickler2014} indicate that
ERA-20C contribution to reanalysis development is the assimilation
of upper-air data from the first half of the 20\textsuperscript{th}
century. This new data processed corresponds to other regions in South
America like Bolivia, Chile and Brazil. Additionally, \citet*{Stickler2014}
argue that most of the current network has been established after
the 1958\textemdash International Geophysical Year\textemdash . However,
it seems that data from the tropical Andes is still not available.

Finally, this section discusses alternative approaches to conduct
fire research in this region. First of all, a higher resolution reanalysis
could be used. For example, \citet*{Field2015} used the NASA Modern
ERA Retrospective-Analysis for Research and Applications (MERRA, \citet{Rienecker2011})
in a recent global fire weather study. This reanalysis product comprises
atmospheric fields with a 1/2\textsuperscript{o} latitude and 2/3\textsuperscript{o}
longitude resolution for the period 1980-2012. Although, daily data
was complete only after 1997. Compiling additional data from neighbouring
stations could be useful to attempt completing missing days. Additionally,
developing a fire activity database is another interesting endeavour
for this region. This task could be accomplished using satellite data
and newspaper reports. Finally, investigating the meteorology of specific
days of extreme fire activity could complement the long-term perspective
analysis. 


\section{Summary}

This chapter described the investigation of fire weather variability
in the Ecuadorian Andes. This research probably is the first scientific
study about fire weather in this region. The study used weather station
data to explore seasonal fire weather from 1997-2012. The investigation
assessed if the McArthur Forest Fire Danger Index (FFDI) is a valid
fire weather metric in Ecuador. Additionally, it examined the observed
fire weather changes in this region. The influence of the El Ni\~no-Southern
Oscillation on fire weather was also explored. Finally, the study
tested if the 20CR and ERA-20C reanalyses can represent fire weather
variability in the Ecuadorian Andes. 

The results show that the FFDI can represent fire weather in the Ecuadorian
Andes. The index yields ``high'' fire danger during July-August-November.
This result is consistent with local reports that identified that
during these months fire activity is higher. Additionally, \textquotedblleft El
Ni\~no\textquotedblright{} events increase fire danger in the Ecuadorian
Andes. Finally, 20CR and ERA-20C do not show skill to capture the
fire weather variability of this region. The next chapter explains
the conclusions from the thesis and proposes future work.

