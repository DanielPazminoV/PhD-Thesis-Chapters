
\chapter{Concluding remarks and recommendations for future work}
\newpage{}

\section{Introduction}

This thesis investigated fire weather in two regions of the Southern
Hemisphere. These study areas were: Victoria, Australia and the Ecuadorian
Andes. Large-scale weather patterns shaped the development of a fire
weather index for Victoria: the ``Victorian Seasonal Bushfire Index''
(VSBI). The research used this index and the McArthur Forest Fire
Danger Index (FFDI) to explore past fire weather changes in this region.
It also tested the VSBI's skill to predict extreme fire weather seasons.
On the other hand, the study assessed the ability of the FFDI to represent
fire weather in Ecuador. Finally, the relationship between fire weather
and the El Ni\~no-Southern Oscillation in the Ecuadorian Andes was examined.

\section{Conclusions}

The major findings of this investigation are:
\begin{enumerate}
\item Bushfire and heatwave weather patterns display differences in Victoria.
Cold fronts define extreme fire weather in this region. In contrast,
persistent excess heat characterises heatwaves. Australia experiences
dry conditions during fire activity in Victoria. On the other hand,
a dry southern Australia contrasts with wet regions during heatwaves.
The patterns show a clear influence of ``El Ni\~no'' events during
bushfires. Conversely, heatwaves in this region are not driven by
this climate mode of variability. 
\item Victoria experienced an increase in fire danger during the period
1974-2010. Observations confirmed this finding using three seasonal
fire weather metrics. Moreover, the most robust reanalysis trend traces
the increase back to 1920. Rising temperatures coupled with a decrease
in rainfall contributed to this long-term tendency.
\item Positive Indian Ocean Dipole (IOD) and ``El Ni\~no'' events precondition
fire activity in Victoria. The main effect of these two drivers is
to reduce the relative humidity in spring. This reduction generates
dry conditions that prepare vegetation for ignitions. The influence
of positive IOD events over fire weather is greater than ``El Ni\~no''
phases in spring. However, the two modes of climate variability are
strongly linked. In summer, ``El Ni\~no'' events become the most important
remote driver of fire weather.
\item A simple index (VSBI) based on local and remote climate drivers shows
skill to forecast fire weather in Victoria. This metric demonstrates
seasonal predictive capability from spring to summer. Additionally,
the VSBI demonstrates comparable accuracy to represent fire weather
than the McArthur Forest Fire Danger Index.
\item The McArthur Forest Fire Danger Index is a useful metric to represent
fire weather in the Ecuadorian Andes. Using this index, the study
provided a quantitative confirmation that the fire danger season in
this region spans from July to November. 
\item ``El Ni\~no'' events increase fire danger in the Ecuadorian Andes.
ENSO influences fire weather in this region through its effect on
temperature and precipitation. In contrast to the coastal response,
the highlands experience rainfall reductions during ``El Ni\~no''.
Simultaneously, temperature increases in the Andes during these events.
A higher forest fire danger during ``El Ni\~no'' is consistent with
its impacts on other Andean natural systems (e.g., glacier retreats).
\end{enumerate}

\section{Future work}

In addition, this thesis identified several aspects that deserve further
investigation:
\begin{enumerate}
\item The exploration of fire weather in Victoria used the 20CR ensemble
mean. The results depicted an increasing trend in fire danger in Victoria
since 1920. However, this reanalysis project has 56 realisations.
Each model output represents an equally possible state of the atmosphere.
Therefore, additional 20R realisations could be used to quantify uncertainties
linked to past fire weather changes.
\item Further research could use the Coupled Model Intercomparison Project
phase 6 (CMIP6) models to project the VSBI. This investigation could
provide insights of future fire weather changes in Victoria.
\item Developing a fire activity database for Ecuador could open further
research opportunities (e.g. land management, environmental impacts,
social studies, etc.). The construction of this database would require using
satellite data and newspaper reports.
\item Investigating the meteorology of specific extreme fire weather days in Ecuador can provide valuable insights for fire emergency services. This analysis could evaluate the synoptic characteristics of fire weather variables using mesoscale model data (e.g. Weather Research and Forecasting Model (WFR)). 
\item The study areas in this investigation included Victoria, Australia
and the Ecuadorian Andes. However, it would be interesting to expand
the areas of investigation. For example, future investigations could
comprise southeast Australia and other Andean regions.
\end{enumerate}


 


