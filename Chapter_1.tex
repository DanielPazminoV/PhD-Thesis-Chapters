
\chapter{Introduction}
\newpage{}

\section{Motivation}

Bushfires cause great human, environmental and economic impacts in
the Southern Hemisphere. This problem is particularly severe in Australia.
\citet{Haynes2010} categorised bushfires as the fourth most lethal
natural hazard in this country. For example, the \textquotedblleft Black
Saturday\textquotedblright{} fires caused 173 fatalities in February
of 2009 \citep{Teague2010}. Additionally, bushfires not only threaten
life but also produce great losses for the Australian economy. \citet{McAneney2009}
reported that bushfires are responsible for 20\% of property losses
in Australia. Indeed, the costs of a single catastrophic bushfire
event can reach billions of dollars (e.g the ``Ash Wednesday'' fires
of 1983 costed \$1.63 billion to insurance companies).

Furthermore, there is evidence that future bushfires could become
more dangerous in several regions of Australia. One of the regions
that might experience greater risk is the south-east of the continent.
\citet{Lucas2007} argues that days with \textquotedblleft very high\textquotedblright{}
and \textquotedblleft extreme\textquotedblright{} fire danger would
become more common in this region. In addition, \citet{Clarke2011}
demonstrated that future fire seasons could start earlier and be more
prolonged. 

The state of Victoria is one the most bushfire prone and vulnerable
areas in southeastern Australia. This region has endured the most catastrophic
bushfire events in Australian history (e.g. ``Black Friday'' (1939),
``Ash Wednesday'' (1983), ``Black Saturday'' (2009)). Additionally,
half of the economic impacts in Australia occurred in Victoria \citep{Luke1978}.
Moreover, climate projections also suggest its fire danger risk will
increase \citep{Clarke2011}. Therefore, Victoria is a region where
preparedness for fire seasons is of utmost importance.

The planning for fire emergencies in Victoria includes the elaboration
of bushfire danger forecasts. These predictions use climate data and
information on the state of the El Ni\~no-Southern Oscillation (ENSO).
Using this information the Australian Bureau of Meteorology generates
seasonal fire weather outlooks. These reports are qualitative. However,
\citet{Harris2013} showed that there is potential to forecast fire
activity quantitatively in Victoria. Thus, further research is necessary
to develop a quantitative forecasting tool for this region. 

Several aspects deserve further investigation to better predict fire
weather in Victoria. For example, understanding the effect of the coincidence of extreme
events such as fire weather and heatwaves. Additionally,
exploring the long-term variability of fire weather could provide
insights on the causes of dangerous fire seasons. It is also important
to better understand the influence of global and regional modes of climate
variability on fire weather.

Extreme bushfire seasons also occur in the Southern Hemisphere in several
regions in South America. For example, bushfires produce severe impacts
to the Ecuadorian Andes ecosystems. In fact, the year 2012 was the
most extreme fire season in the recent history \citep{MinisteriodelAmbiente2013,SecretariadeAmbiente2013}.
During this season bushfires destroyed more than 21,570 Ha \citep{MinisteriodelAmbiente2013}.
These fires affected national parks in one of the most biodiverse
countries in the world. Despite these losses, there is a lack of understanding
of how weather drives fire events in Ecuador. This study is perhaps the first
investigation about fire weather in this region. 

South America and Australia share common aspects linked to fire activity.
For example, modes of climate variability such as El Ni\~no-Southern
Oscillation (ENSO) influence fire weather in both continents \citep{Williams1999,Chen2011}.
This research further examines this relationship on a regional scale.
The investigation focus on two study areas: Victoria, Australia and
the Ecuadorian Andes. These two regions also share the \textit{Eucalyptus}
as the characteristic fire-prone vegetation. The \textit{Eucalyptus} are Australian indigenous species
that were introduced in Ecuador in 1865 to foster the forestry industry \citep{Anchaluisa2013}.    


\section{Aim and scope}

The aim of this thesis is to better understand the drivers and evolution
of fire weather in two regions of the Southern Hemisphere: Victoria,
Australia and the Ecuadorian Andes. Specifically, this study will
address the following research questions:
\begin{enumerate}
\item Co-occurrence of extreme events in Victoria: Fire weather spatial
patterns and their relationship with heatwaves. 


\textbullet{} Some heatwave events cause bushfires while others do
not. Can synoptic weather patterns explain the difference between
these two types of events? 


\textbullet Can fire weather spatial patterns be used to develop a
fire danger index for Victoria?

\item Long-term fire weather changes in Victoria and the Ecuadorian Andes.


\textbullet How did the variability of fire weather in Victoria behave
over the last century?


\textbullet{} What was the observed fire weather variability in the
Ecuadorian Andes?

\item Influence of ENSO and other climate drivers on fire weather in Victoria
and the Ecuadorian Andes.


\textbullet{} How do the most important modes of climate variability
affect fire weather in Victoria?


\textbullet{} Can remote climate drivers increase the skill of forecasts
of extreme fire weather in Victoria? 


\textbullet{} What is the relationship between El Ni\~no-Southern Oscillation
and fire weather in the Ecuadorian Andes?

\end{enumerate}

\section{Thesis outline}

Following this introduction, Chapter 2 provides background information
to understand the thesis. It defines concepts as well as characterises
the study areas. Chapter 3 reviews the scientific literature
linked to this project. This part of the thesis also identifies the
research gaps addressed in the investigation. Chapter 4 aims to answer
the first research question. This chapter investigates the relationship
between bushfire and heatwave weather patterns in Victoria, Australia.
Subsequently, Chapter 5 builds on this work to develop a new fire
weather index for Victoria. Additionally, this chapter investigates
the long-term variability of fire weather in Victoria. Chapter 5 also
examines climate-bushfire relationships for seasonal forecasting purposes.
Chapter 6 reproduces the results for Victoria using alternative data
sources and definitions to assess the confidence in the results of Chapter 5. 
Chapter 7 uses the methods applied in previous
chapters to investigate fire weather in Ecuador. Chapter 8 highlights
the conclusions of the thesis and recommends future
research derived from this work.
