
\chapter{Reproducibility of results for Victoria}
\newpage{}


\section{Introduction}

Chapter 4 and 5 investigated complementary aspects of fire weather
in Victoria, Australia. Chapter 4 examined the differences between
bushfire and heatwave weather patterns. On the other hand, Chapter
5 analysed two additional aspects fire weather in Victoria. It explored
the long-term variability and forecasting skill of fire weather in
this region. Chapter 6 aims to test if the findings of these two chapters
are robust. The investigation replicates these analyses using the
same methods with alternative data sources. It also consolidates the
discussion about fire weather in Victoria.

Four sections comprise this chapter. The data and methodology section
provides details about the alternative data sources. This section
also summarises the methods described in the previous chapters. However,
the reader has references to the corresponding chapters for further
details about the methodologies. The results section reproduces Chapter's
4 and 5 findings. The discussion section provides an integrated analysis
comparing the original and reproduced results. The summary gives an
overview of the chapter and highlights the key findings.


\section{Data and methodology}


\subsection{Data}


\subsubsection{Bushfire activity records}

The bushfire events databses are those used in the previous chapters.
The two datasets were produced by the Department of Environment, Land,
Water and Planning (DELWP) of Victoria and Risk Frontiers (RF). Both
datasets are comparable using events greater than 10,000 Ha as the
threshold to define bushfire events. The selected period of analysis
using this definition was 1961-2010. The datasets have a greater correlation
during this time. Chapter 4 discusses further details about these
databases.


\subsubsection{Heatwave dates}

The replication of the experiment conducted in Chapter 4 uses an alternative
heatwave definition. This chapter discusses the results using the
heatwaves dates computed using \citet{Pezza2012} definition. 


\subsubsection{Fire weather metrics}

The fire weather metrics are those applied in the previous chapter.
The analyses use two seasonal metrics of the McArthur Forest Fire
Danger Index (FFDI). The first FFDI metric is a cumulative sum of
daily FFDI values (FFDI\textsubscript{cum}). The second FFDI metric
is the number of days with FFDI greater than 25 (FFDI\textsubscript{>25}).
This metric represents days with a fire danger rating ranging from
``very high'' to ``catastrophic''. Additionally, an alternative
metric used was the ``Victorian Seasonal Bushfire Index'' (VSBI)
proposed in Chapter 5. Chapters 2 and 5 provide details on the formulation
of the FFDI and VSBI respectively. Chapter 3 presents a critical review
of the FFDI. Chapter 5 discussed the limitations of the VSBI. This
chapter further discusses these indices.


\subsubsection{Climate variables}

The replication of bushfire and heatwave weather patterns used the
same climate data as in Chapter 4. The climate data used in this chapter
was the Twentieth Century Reanalysis Project version 2 (20CR), the
Hadley Centre Sea Ice and Sea Surface Temperature data set (HadISST1),
and the Australian Climate Observations Reference Network-Surface
Air Temperature (ACORN-SAT). Chapter 4 specifies the characteristics
of these datasets. 

On the other hand, reproducing Chapter 5 results required an alternative
source of reanalysis data. During the execution of this research,
the European Centre for Medium-Range Weather Forecasts (ECMWF) launched
the European Reanalysis of Global Climate Observations ERA-Clim project
\citep{Stickler2014,Stickler2014a}. This investigation used one of
its products: the ERA-20C twentieth-century reanalysis. ERA-20C, assimilated
surface pressure data\textemdash as well as marine wind observations\textemdash and
used sea surface temperatures and sea ice concentrations as boundary
conditions. It comprises a dataset that provides 3-hourly data, with
125 Km spatial resolution and 92 vertical levels. 

The use of the ERA-20C dataset has several advantages for this research
project. First of all, it provides data with higher resolution. ERA-20C
comprises data with a 125 Km spatial resolution, while the 20CR outputs
only reach 200 Km. Secondly, the ERA-20C data yield surface\textemdash 10m\textemdash climate
data. On the other hand, version 2 of the 20CR\textemdash which was
used when the project started\textemdash only provides values at the
1000 hPa level. Therefore, ERA-20C required less bias-corrections.
Finally, ERA-20C provides a dataset to assess the results independently.
Even though alternative ensemble members of the 20CR could have been
used to reproduce the results, the use of a different reanalysis product
provides an independent source of comparison.

On the other hand, ERA-20C also has limitations. This reanalysis product
yields data from 1900 to 2010, while the 20CR project covers a larger
period (1851-2014). Nevertheless, the analyses conducted in the previous
chapters do not use years before 1900. Therefore, this limitation
is not a disadvantage for this project. Additionally, ERA-20C is a
deterministic model. It provides only one ensemble of data. In contrast,
the 20CR yields 56 realisations of possible states of the climate.
This investigation has only used the ensemble mean of the 20CR. Thus,
the deterministic nature of the ERA-20CR is not a disadvantage either.


\subsubsection{Climate indices}

This chapter uses the same climate indices analysed in previous chapters.
These include El Ni\~no-Southern Oscillation (ENSO) indices such as
NINO3.4, NINO1+2, and Southern Oscillation Index (SOI). It also uses
the Dipole Mode Index (DMI) which quantifies the intensity of Indian
Ocean Dipole (IOD) events. Chapter 2 provides further details about
ENSO and IOD and their indices.


\subsection{Methodology}


\subsubsection{Composite analysis}

The composite analysis reproduces the results obtained in Chapter
4. These computations facilitate the comparison of bushfire and heatwave
weather patterns. The calculations included daily and seasonal composites
spanning the period 1961-2010. The daily plots compare weather patterns
before and after (from day minus 1 to day plus one) of the events.
The seasonal composites display the differences during the antecedent
(September-October-November) and concurrent (December-January-February)
seasons. 

For this test, the bushfire composites used the Risk Frontiers bushfire
database (RF). On the other hand, the criteria defined by \citet{Pezza2012}
is adopted as a second heatwave definition. The analyses used these
data to test the sensitivity of the results. The analysis excluded
days in which bushfire and heatwaves occurred simultaneously (see
Appendix A for list of dates). This procedure allowed producing plots
that represent each type of event exclusively. Chapter 4 further describes
the composite analysis methodology.


\subsubsection{Fire weather variability analysis}

\LyXZeroWidthSpace This section replicates the results produced in
Chapter 5. Fire weather variability was explored using seasonal metrics
of the FFDI as well as the VSBI. For this experiment, the calculations
used the ERA-20C as an alternative source of data. The first step
in this process was to compare the reanalysis data against observations.
This comparison used probability density functions and seasonal time
series to evaluate if any bias-corrections were required. Initially,
the climate data analysis was conducted for the period of available
weather station data (1974-2010). However, time series were also plotted
for the period 1920-2010 using the reanalysis data.

Subsequently, the computation of the FFDI (FFDI\textsubscript{cum}
and FFDI\textsubscript{>25}) and VSBI used these climate data. These
computations analysed fire weather variability during the fire season
DJF. These calculations used observations and reanalysis data for
the period 1974-2010. The analysis for the period 1920-2010 represented
fire weather using reanalysis data only. Chapter 5 describes more
details of the methodology to analyse fire weather variability.


\subsubsection{Seasonal forecasting skill analysis}

This section reproduces the findings for seasonal forecasting skill
in Victoria (see Chapter 5). In this analysis, linear correlations
revealed the relationship between four groups of data. The data comprised
weather variables, climate indices, fire weather metrics and fire
activity records. The exploration of climate-bushfire relationships
comprised the seasons September-October-November (SON-SON), December-January-February
(DJF-DJF), and from spring to summer (SON-DJF). The analysis computed
the correlations for each season during three periods : 1974-2010,
1961-2010 and 1920-2010. The first period includes correlations using
weather station data. The second period covers the entire length of
bushfire activity records. The last period spans most of the reanalysis
dataset coverage. Chapter 5 describes finer aspects of the methodology
for this analysis.


\section{Results}

The results section comprises three parts. The first sub-section displays
the results of bushfire and heatwave weather patterns. The next sub-section
shows the fire weather variability results. The last section analyses
seasonal climate-bushfire relationships and fire weather forecasting
skill.


\subsection{Bushfire and heatwave weather patterns}

This sub-section aims to reproduce the results of bushfire and heatwave
weather patterns. It shows daily and seasonal composites using an
alternative bushfire database. For this experiment, Risk Frontiers
provided the bushfire records (RF). Additionally, the investigation
used a second heatwave definition. \citet{Pezza2012} provide a detailed
explanation of the heatwave definition adopted. This sub-section starts
displaying time series of fire activity and heatwaves. These time
series are useful to understand the evolution of these two types of
events. 


\subsubsection{Fire activity and heatwave events time series}

Figure \ref{fig:Time series of bushfires, heatwaves and Southern Oscillation Index (SOI) in Victoria, Australia during the period 1961-2011 (r)}
shows a time series of bushfire and heatwave events in Victoria during
the period 1961-2010. Panel a) shows the evolution of extreme bushfire
events in this region using the RF database. According to its records,
there is no evident trend in the occurrence of bushfires during this
period. As expected, they seem to peak during very strong negative
phases of the Southern Oscillation Index (shown in blue lines). However,
the long-term relationship shows a rather weak relationship (r=-0.33).
On the other hand, panel b) displays a time series of heatwaves using
\citet{Pezza2012} definition during the same period. The panel shows
that extreme heatwaves occur less frequently than extreme fire events.
Additionally, there is no correlation between the heatwave and SOI
time series (r=-0.04). 

\begin{figure}[h]
\noindent \begin{centering}
\includegraphics[scale=0.7]{Chapter_6/Figures/Fire_activity_and_heatwave_frequencies/BF_HW_frequencies}
\par\end{centering}

\caption[Time series of bushfires, heatwaves and Southern Oscillation Index
(SOI) in Victoria, Australia during the period 1961-2011]{Time series of bushfires, heatwaves and Southern Oscillation Index
(SOI) in Victoria, Australia during the period 1961-2011. Panel a)
shows 41 bushfire events and b) 13 heatwaves in the period 1961-2011
in Victoria, Australia (red bars representing events for the season
December-January-February). Southern Oscillation Index (SOI) time
series is presented as the average for the season December-January-February
(blue lines). Monthly SOI values taken from the Bureau of Meteorology
web page. The bushfire, heatwave and SOI time series were fitted to
a 6\protect\textsuperscript{th} order polynomial curve (black lines).
R coefficients of bushfire and heatwave events with the SOI are -0.33
and -0.04 respectively. Bushfire events data were obtained from the
Risk Frontiers database. Heatwaves were defined using \citet{Pezza2012}
heatwave definition. \label{fig:Time series of bushfires, heatwaves and Southern Oscillation Index (SOI) in Victoria, Australia during the period 1961-2011 (r)}}


\end{figure}



\subsubsection{Weather patterns for daily climatology}

The plots show the evolution of the synoptic patterns from day -1
to day +1 for bushfire and heatwave days. Panels a), b), and c) show
the composites for bushfires. On the other hand, panels d), e) and
f) display the heatwave counterparts. The anomalous areas statistically
significant at the 5\% level are marked with ``+''. The figures
show composites for air temperature, relative humidity and geopotential
height.

Figure \ref{fig:Air temperature anomalies at 1000 hPa level for day -1 to day +1 respectively of 27 bushfire events (period 1961=0020132011) (r)}
shows daily composites of air temperature at the 1000 hPa level. Bushfires
exhibit the greatest region with positive anomalies (+ 3 \textsuperscript{o}C)
during day -1. The computation considered fire records only in Victoria.
However, the hot anomalies influence several regions in Australia.
It directly affects Victoria and Southern Australia; as well as sectors
of West Australia, New South Wales, the Northern Territory and the
Bass Strait. During this day, Queensland and the coastal West Australia
display cold anomalies. Extreme bushfires start (day 0) when a cold
front reduces the hot anomalous region. A day after the cold front
passage, warm anomalies persist in Victoria, New South Wales and the
Northen Territory.

On the other hand, heatwaves display persistent hot anomalies (+ 5
\textsuperscript{o}C) during the three days. During day -1, heatwaves
show more intense anomalies than bushfires. The extent of the region
affected by hot anomalies is also greater. This area even includes
the coastal area of West Australia. Heatwaves and bushfires present
cold anomalies in Queensland a day before the start of the events.
When heatwaves start (day 0), the hot anomalies remain without substantial
change. In contrast, the cold anomalies in Queensland become slightly
stronger. Additionally, the coastal region of Western Australia starts
showing cold anomalies as well. During the second day of heatwaves
(day +1), the hot anomalies persist. However, there is a slight increase
of cold anomalies in Western Australia.

\begin{figure}[h]
\noindent \begin{centering}
\includegraphics[scale=0.65]{Chapter_6/Figures/Weather_patterns/T_daily}
\par\end{centering}

\caption[Air temperature daily anomalies at 1000 hPa level during extreme bushfire
and heatwave events in Victoria, Australia during the period 1961-2010]{Air temperature daily anomalies at 1000 hPa level during extreme bushfire
and heatwave events in Victoria, Australia during the period 1961-2010.
Panels a), b), c), d), e) and f) show anomalies for day -1 to day
+1 of 41 bushfire and 13 heatwave events respectively. Bushfire records
sourced from Risk Frontiers (RF) database. Heatwaves computed using
\citet{Pezza2012} criteria. Climate data from the Tweentieth Century
Reanalysis Project (20CR). Areas marked with ``+'' show anomalies
statistically significant at the 5\% level. Cold fronts and persistent
excess heat characterise extreme fire weather and heatwaves weather
respectively.\label{fig:Air temperature anomalies at 1000 hPa level for day -1 to day +1 respectively of 27 bushfire events (period 1961=0020132011) (r)}}
\end{figure}
 

Figure \ref{fig:Relative humidity anomalies at 1000 hPa level for day -1 to day +1 respectively of 27 bushfire events (period 1961=0020132011) (r)}
depict daily composites for relative humidity at the 1000 hPa level.
One day before bushfires start, Victoria shows strong negative anomalies
(-15\%). In contrast, the coastal region of Western Australia displays
wet anomalies. A strong cold front characterises the start of bushfires
(day 0) reducing the area affected by dry anomalies. Although, Victoria
remains extremely dry. On day +1, the cold front brings wet anomalies
to most of Australia. However, Victoria and New South Wales still
show dry anomalies (-5\%). 

Relative humidity anomalies during heatwaves do not display significant
changes. During day -1, strong negative anomalies (-15\%) affect South
Australia. Although, this dry pattern also influences Victoria. Weak
wet anomalies surround the overall dry pattern. At day zero, dry anomalies
concentrate in Southern Australia and Victoria. In contrast, wet anomalies
on this day become stronger. During the second day of heatwaves (day
+1), the synoptic pattern resembles substantially the pattern one
day before.

\begin{figure}[h]
\noindent \begin{centering}
\includegraphics[scale=0.65]{Chapter_6/Figures/Weather_patterns/RH_daily}
\par\end{centering}

\caption[Relative humidity daily anomalies at 1000 hPa level during extreme
bushfire and heatwave events in Victoria, Australia during the period
1961-2010]{Relative humidity daily anomalies at 1000 hPa level during extreme
bushfire and heatwave events in Victoria, Australia during the period
1961-2010. Panels a), b), c), d), e) and f) show anomalies for day
-1 to day +1 of 41 bushfire and 13 heatwave events respectively. Bushfire
records sourced from Risk Frontiers (RF) database. Heatwaves computed
using \citet{Pezza2012} criteria. Climate data from the Tweentieth
Century Reanalysis Project (20CR). Areas marked with ``+'' show
anomalies statistically significant at the 5\% level. Southern Australia
is dry during bushfires in Victoria while the continent displays intense
dry-wet contrasts linked to heatwaves in this region. \label{fig:Relative humidity anomalies at 1000 hPa level for day -1 to day +1 respectively of 27 bushfire events (period 1961=0020132011) (r)}}
\end{figure}


Figure \ref{fig: Geopotential height anomalies at 500 hPa level for day -1 to day +1 respectively of 27 bushfire events (period 1961=0020132011) (r)}
illustrate daily anomaly composites for geopotential height at the
500 hPa level. Bushfires exhibit a clear wave train pattern. One day
before bushfires are triggered, Victoria shows strong and positive
anomalies (+60 m). These anomalies emerge in the Antartic region.
Overall, the intensity of the frontal activity characterises the pattern. 

Heatwaves also display a wave train pattern with intense anomalies.
In contrast to the fire weather pattern, the Antartic region shows
areas with negative anomalies. The synoptic pattern remains constant
before and after the start of heatwaves. 

\begin{figure}[h]
\noindent \begin{centering}
\includegraphics[scale=0.65]{Chapter_6/Figures/Weather_patterns/Hgt_daily}
\par\end{centering}

\caption[Geppotential height daily anomalies at 1000 hPa level during extreme
bushfire and heatwave events in Victoria, Australia during the period
1961-2010]{Geppotential height daily anomalies at 1000 hPa level during extreme
bushfire and heatwave events in Victoria, Australia during the period
1961-2010. Panels a), b), c), d), e) and f) show anomalies for day
-1 to day +1 of 41 bushfire and 13 heatwave events respectively.Bushfire
records sourced from Risk Frontiers (RF) database. Heatwaves computed
using \citet{Pezza2012} criteria. Climate data from the Tweentieth
Century Reanalysis Project (20CR). Areas marked with ``+'' show
anomalies statistically significant at the 5\% level. Bushfires display
intense frontal activity while heatwaves show persistent anomalies.\label{fig: Geopotential height anomalies at 500 hPa level for day -1 to day +1 respectively of 27 bushfire events (period 1961=0020132011) (r)}}
\end{figure}



\subsubsection{Seasonal large-scale circulation patterns }

The figures in this sub-section show panels a) with bushfire seasons
weather patterns. On the other hand, panels b) presents the results
for heatwaves. Black dots show areas with statistical significance
at the 5\% level. The composites include sea surface temperature,
air temperature, and relative humidity.

Figure \ref{fig:Sea surface temperature anomalies of bushfire and heatwave seasons in December-January-February for the period 1961=0020132011 (r)}
shows a composite of sea surface temperature anomalies during the
season DJF. The patterns are consistent with the findings in Chapter
4. Panel a) shows strong warming anomalies in the tropical Pacific
Ocean. In contrast, panel b) suggests a different dynamic. During
heatwave events the most relevant feature in the composite is the
large area of warming anomalies over the southern Indian Ocean. North
to this region, a small area of cool anomalies contrasts the overall
warming pattern. 

\begin{figure}[h]
\noindent \begin{centering}
\includegraphics[scale=0.75]{Chapter_6/Figures/Weather_patterns/SSTs}
\par\end{centering}

\caption[Sea surface temperature (SST) anomalies composite of a) 17 bushfire
and b) 13 heatwave seasons occurred in the season December-January-February
(DJF) from 1961\textendash 2011]{Sea surface temperature (SST) anomalies composite of a) 17 bushfire
and b) 13 heatwave seasons occurred in the season December-January-February
(DJF) from 1961\textendash 2011. Bushfire records sourced from Risk
Frontiers (RF) database. Heatwaves were computed using \citet{Pezza2012}
criteria. Climate data obtained from the Met Office Hadley Centre\textquoteright s
sea ice and sea surface temperature dataset (HadISST1) dataset. Shaded
areas show anomalies statistically significant at the 5\% level. Extreme
fire seasons in Victoria show a clear ``El Ni\~no'' pattern while
heatwaves display a neutral condition. \label{fig:Sea surface temperature anomalies of bushfire and heatwave seasons in December-January-February for the period 1961=0020132011 (r)}}


\end{figure}


Figure \ref{fig:Temperature anomalies (1000 hPa level) of bushfire and heatwave seasons in September-October-November for the period 1961=0020132011 (r)}
depicts the seasonal composites for air temperature at the 1000 hPa
level during SON. During the months immediately before extreme bushfire
seasons, Victoria is a ``hotspot''. This region shows statistically
significant hot anomalies (+ 0.5 \textsuperscript{o}C). The rest
of the Australian continent also displays warm anomalies. Although
these anomalies are less intense and are not statistically significant.
The tropical Pacific Ocean shows near surface warm anomalies. Nevertheless,
the pattern is not statistically significant either.

On the other hand, heatwaves show less intense anomalies in Victoria.
The spring composite displays no clear patterns in the tropical Pacific
Ocean. However, the Indian Ocean reflects a stronger north-south contrast
between hot and cool anomalies.

\begin{figure}[h]
\noindent \begin{centering}
\includegraphics[scale=0.75]{Chapter_6/Figures/Weather_patterns/T_son}
\par\end{centering}

\caption[Seasonal composite for air temperature (T) anomalies (1000 hPa level)
for a) 17 bushfire and b) 13 heatwave seasons occurred in the season
September-October-November (SON) from 1961\textendash 2011]{Seasonal composite for air temperature (T) anomalies (1000 hPa level)
for a) 17 bushfire and b) 13 heatwave seasons occurred in the season
September-October-November (SON) from 1961\textendash 2011. Bushfire
records sourced from Risk Frontiers (RF) database. Heatwaves were
computed using \citet{Pezza2012} criteria. Climate data obtained
from the Tweentieth Century Reanalysis Project (20CR). Shaded areas
show anomalies statistically significant at the 5\% level. Extreme
fire seasons in Victoria show an ``El Ni\~no-like'' pattern while
heatwaves display a neutral condition.\label{fig:Temperature anomalies (1000 hPa level) of bushfire and heatwave seasons in September-October-November for the period 1961=0020132011 (r)}}
\end{figure}


Figure \ref{fig:Temperature anomalies (1000 hPa level) of bushfire and heatwave seasons in December-January-February for the period 1961=0020132011 (r)}
shows the composites for air temperature at the 1000 hPa level for
the season DJF. During extreme bushfire seasons, the Australian continent
is warm (+ 0.5 \textsuperscript{o}C). Additionally, the ``Ni\~no-like''
pattern is clearly defined in the NINO3.4 region. Although, it is
not intense in the long-term average. In contrast, seasons with extreme
heatwave events display hot anomalies only in Victoria and Western
Australia. In fact, most of the Australian continent displays cold
anomalies. The Tasman Sea region also shows near-surface negative
anomalies. 

\begin{figure}[h]
\noindent \begin{centering}
\includegraphics[scale=0.75]{Chapter_6/Figures/Weather_patterns/T_djf}
\par\end{centering}

\caption[Seasonal composite for air temperature (T) anomalies (1000 hPa level)
for a) 17 bushfire and b) 13 heatwave seasons occurred in the season
December-January-February (DJF) from 1961\textendash 2011]{Seasonal composite for air temperature (T) anomalies (1000 hPa level)
for a) 17 bushfire and b) 13 heatwave seasons occurred in the season
December-January-February (DJF) from 1961\textendash 2011. Bushfire
records sourced from Risk Frontiers (RF) database. Heatwaves were
computed using the \citet{Pezza2012} criteria. Climate data obtained
from the Tweentieth Century Reanalysis Project (20CR). Shaded areas
show anomalies statistically significant at the 5\% level. Australia
is a ``hot'' continent during extreme fire seasons in Victoria while
heatwaves display hot-cool contrasts.\label{fig:Temperature anomalies (1000 hPa level) of bushfire and heatwave seasons in December-January-February for the period 1961=0020132011 (r)}}
\end{figure}


Figure \ref{fig:Relative humidity anomalies (1000 hPa level) of bushfire and heatwave seasons in September-October-November for the period 1961=0020132011 (r)}
displays the composites for relative humidity at the 1000 hPa level
during the season SON. This variable shows that Victoria\textemdash and
Southeastern Australia\textemdash is drier than average in the months
preceding extreme bushfire seasons (-1.5\%). Other regions in Australia
also exhibit negative anomalies with less intensity. During heatwaves
Victoria and Western Australia are also dry. However, most of the
continent shows wet conditions. 

\begin{figure}[h]
\noindent \begin{centering}
\includegraphics[scale=0.75]{Chapter_6/Figures/Weather_patterns/RH_SON}
\par\end{centering}

\caption[Seasonal composite for relative humidity (RH) anomalies (1000 hPa
level) for a) 17 bushfire and b) 13 heatwave seasons occurred in the
season September-October-November (SON) from 1961\textendash 2011]{Seasonal composite for relative humidity (RH) anomalies (1000 hPa
level) for a) 17 bushfire and b) 13 heatwave seasons occurred in the
season September-October-November (SON) from 1961\textendash 2011.
Bushfire records sourced from Risk Frontiers (RF) database. Heatwaves
were computed using \citet{Pezza2012} criteria. Climate data obtained
from the Tweentieth Century Reanalysis Project (20CR). Shaded areas
show anomalies statistically significant at the 5\% level. Australia
is dry during the antecedent fire season in Australia is dry during
the antecedent fire season in Victoria while wet conditions prevail
in the continent linked to heatwaves in this region.\label{fig:Relative humidity anomalies (1000 hPa level) of bushfire and heatwave seasons in September-October-November for the period 1961=0020132011 (r)}}
\end{figure}


Figure \ref{fig:Relative humidity anomalies (1000 hPa level) of bushfire and heatwave seasons in September-October-November for the period 1961=0020132011 (r)}
shows the seasonal composites for relative humidity at the 1000 hPa
level in DJF. During extreme bushfire seasons, almost every region
in Australia is dry (from -0.5\% to 1.5\%). Interestingly, during
seasons that experience extreme heatwaves, less areas display dry
anomalies (e.g. Victoria and Western Australia). Surprisingly, most
of the continent shows wet conditions (+ 1.5\%). 

\begin{figure}[h]
\noindent \begin{centering}
\includegraphics[scale=0.75]{Chapter_6/Figures/Weather_patterns/RH_DJF}
\par\end{centering}

\caption[Seasonal composite for relative humidity (RH) anomalies (1000 hPa
level) for a) 17 bushfire and b) 13 heatwave seasons occurred in the
season December-January-February from 1961\textendash 2011]{Seasonal composite for relative humidity (RH) anomalies (1000 hPa
level) for a) 17 bushfire and b) 13 heatwave seasons occurred in the
season December-January-February from 1961\textendash 2011. Bushfire
records sourced from Risk Frontiers (RF) database. Heatwaves were
computed using \citet{Pezza2012} criteria. Climate data obtained
from the Tweentieth Century Reanalysis Project (20CR). Shaded areas
show anomalies statistically significant at the 5\% level. In the
concurrent fire season in Victoria Australia is dry during while during
heatwaves most of the continent is wet (except Southern Australia).\label{fig:Relative humidity anomalies (1000 hPa level) of bushfire and heatwave seasons in December-January-February for the period 1961=0020132011 (r)}}
\end{figure}



\subsection{Seasonal fire weather variability}


\subsubsection{McArthur Forest Fire Danger computation (1974-2010)}

The analysis of fire weather variability started with the computation
of the FFDI. The computation used weather station data as well as
an alternative reanalysis dataset (ERA-20C). The period of analysis
was 1974-2010. Weather station data was available only for this period.
The FFDI results using weather station (WS) and reanalysis data (ERA-20C)
were compared using probability density functions (PDFs) and time
series. Figure \ref{fig:Probability density functions of orginal and bias-corrected Forest Fire Danger Index results for Victoria, Australia during the period 1974-2010 (r)}
shows that the distributions of the FFDI computed with reanalysis
and weather station data are similar. The same procedure was applied
to each FFDI variable. Additionally, Figure \ref{fig:Time series of fire weather variables for the season December-January-February in Victoria, Australia during the period 1974-2010 (r)}
displays the comparisons using original and bias-corrected data to
plot seasonal time series. These steps revealed that precipitation
could be bias-corected using linear scaling techniques. 

\begin{figure}[h]
\noindent \begin{centering}
\includegraphics[scale=0.75]{Chapter_6/Figures/Fire_weather_variability/FFDI/1974-2010/PDFs/VIC_FFDI_ERA-20C}
\par\end{centering}

\caption[Probability density functions of orginal and bias-corrected Forest
Fire Danger Index (FFDI) results for Victoria, Australia during the
period 1974-2010]{Probability density functions of orginal and bias-corrected Forest
Fire Danger Index (FFDI) results for Victoria, Australia during the
period 1974-2010. The plots shows the adjustment in the distribution
of FFDI data. Data source: European Reanalysis of Global Climate Observations
(ERA-20C) and Weather Stations in Victoria (WS). \label{fig:Probability density functions of orginal and bias-corrected Forest Fire Danger Index results for Victoria, Australia during the period 1974-2010 (r)}}
\end{figure}


Figure \ref{fig: Cumulative Forest Fire Danger Index time series during December-January-February in Victoria, Australia for the period 1974-2010 (r)}
shows time series of seasonal fire weather using the cumulative FFDI
metric. The figure displays the results using original and bias-corrected
data. The comparison with original data demonstrates a moderate correlation
with observations (r=0.52, p=0.00). This figure also shows that the
bias-correction decreases the magnitude of the results. 

\begin{figure}[h]
\noindent \begin{centering}
\includegraphics[scale=0.75]{Chapter_6/Figures/Fire_weather_variability/FFDI/1974-2010/Time_series/FFDI_cum}
\par\end{centering}

\caption[Cumulative Forest Fire Danger Index (FFDI\protect\textsubscript{cum})
time series during December-January-February (DJF) in Victoria, Australia
for the period 1974-2010]{Cumulative Forest Fire Danger Index (FFDI\protect\textsubscript{cum})
time series during December-January-February (DJF) in Victoria, Australia
for the period 1974-2010. Climate data obtained from the European
Reanalysis of Global Climate Observations (ERA-20C, ERA-20C(bc) respectively)
and weather stations in Victoria (WS). The correlation coefficients
between weather station and reanalysis data are a) r=0.52, p=0.00,
n=37, and b) r=0.51, p=0.00, n=37. \label{fig: Cumulative Forest Fire Danger Index time series during December-January-February in Victoria, Australia for the period 1974-2010 (r)} }
\end{figure}


Figure \ref{fig:Number of days with Forest Fire Danger Index greater than 25 time series during December-January-February in Victoria, Australia for the period 1974-2010 (r)}
shows the results for an alternative seasonal fire weather variability
metric. Fire weather variability is expressed as the number of days
with FFDI greater than 25. The results show that using this metric
there is less agreement between WS and ERA-20C (r=0.37, p=0.03). This
process also reduced the FFDI time series magnitude to values comparable
to observations.

\begin{figure}[h]
\noindent \begin{centering}
\includegraphics[scale=0.65]{Chapter_6/Figures/Fire_weather_variability/FFDI/1974-2010/Time_series/FFDI_g25}
\par\end{centering}

\noindent \centering{}\caption[umber of days with Forest Fire Danger Index greater than 25 (FFDI\protect\textsubscript{>25})
time series during December-January-February (DJF) in Victoria, Australia
for the period 1974-2010]{Number of days with Forest Fire Danger Index greater than 25 (FFDI\protect\textsubscript{>25})
time series during December-January-February (DJF) in Victoria, Australia
for the period 1974-2010. Climate data obtained from the European
Reanalysis of Global Climate Observations (ERA-20C, ERA-20C(bc) respectively)
and weather stations in Victoria (WS). The correlation coeficients
between weather station and reanalysis data are a) r=0.37, p=0.03,
n=37, and b) r=0.52, p=0.00, n=37. \label{fig:Number of days with Forest Fire Danger Index greater than 25 time series during December-January-February in Victoria, Australia for the period 1974-2010 (r)}}
\end{figure}


Figures \ref{fig:Probability density functions of fire weather variables in Victoria, Autralia for the period 1974-2010 (r)}
and \ref{fig:Time series of fire weather variables for the season December-January-February in Victoria, Australia during the period 1974-2010 (r)}
displays fire weather variables PDFs and time series respectively.
These figures compare WS and original reanalysis data (ERA-20C). Each
figure shows four panles. The panels show graphs for maximum temperature
(panel a), relative humidity (panel b), wind speed (panel c) and precipitation
(panel d). Figure \ref{fig:Probability density functions of fire weather variables in Victoria, Autralia for the period 1974-2010 (r)}
shows that most variables\textemdash with the exception of precipitation\textemdash have
a similar distribution of data. 

\begin{figure}[h]
\noindent \begin{centering}
\includegraphics[scale=0.65]{Chapter_6/Figures/Fire_weather_variability/FFDI_variables/PDFs_variables}
\par\end{centering}

\caption[Probability density functions of fire weather variables in Victoria,
Autralia for the period 1974-2010]{Probability density functions of fire weather variables in Victoria,
Autralia for the period 1974-2010. Panels a), b), c) and d) display
the functions for maxium temperature, relative humidity, wind speed
and precipitation respectively. Data source: European Reanalysis of
Global Climate Observations (ERA-20C) and weather stations in Victoria
(WS). Most variables\textemdash with the exception of precipitation\textemdash have
a similar distribution of data \label{fig:Probability density functions of fire weather variables in Victoria, Autralia for the period 1974-2010 (r)}}


\end{figure}


\begin{figure}[h]
\noindent \begin{centering}
\includegraphics[scale=0.75]{Chapter_6/Figures/Fire_weather_variability/FFDI_variables/TS_variables}
\par\end{centering}

\caption[Time series of fire weather variables for the season December-January-February
(DJF) in Victoria, Australia during the period 1974-2010]{Time series of fire weather variables for the season December-January-February
(DJF) in Victoria, Australia during the period 1974-2010. Panels a),
b), c) and d) display time series for maximum temperature, relative
humidity, wind speed and precipitation respectively. The correlation
coefficients between observations and reanalysis data are a) r=0.63,
p<0.05, b) r=0.65, p<0.05, c) r=0.35, p<0.05, d) r=0.49, p<0.05 .
Data source: European Reanalysis of Global Climate Observations (ERA-20C)
and weather stations in Victoria (WS). Precipitation shows the largest
deviation from the mean (factor of 3.08). \label{fig:Time series of fire weather variables for the season December-January-February in Victoria, Australia during the period 1974-2010 (r)}}


\end{figure}


Figures \ref{fig: Precipitation probability density functions for Victoria, Australia during the period 1974-2010 (r)}
and \ref{fig:Precipitation time series for the season December-January-February in Victoria during the period 1974-2010}
show PDFs and time series for precipitation. Panels a) show the comparison
bewteeen weather station and orignal reanalysis data. On the other
hand, panel b) show the reuslts of the bias-correction process. Reanalysis
data overestimates rainfall in Victoria. However, ERA-20C moderately
captures variability of precipitation (r=0.49,p=0.00). Therefore,
the precipitation values were bias-corrected using multiplicative
linear scaling (factor of 3.08). As shown in Figure \ref{fig:Precipitation time series for the season December-January-February in Victoria during the period 1974-2010}
(panel b), the bias-correction reduced the magnitude of the ERA-20C
data to values comparable to observations. 

\begin{figure}[h]
\noindent \begin{centering}
\includegraphics[scale=0.75]{Chapter_6/Figures/Fire_weather_variability/FFDI_variables/P_pdfs}
\par\end{centering}

\caption[Precipitation probability density functions for Victoria, Australia
during the period 1974-2010]{Precipitation probability density functions for Victoria, Australia
during the period 1974-2010. Data source: Original and bias-corrected
European Reanalysis of Global Climate Observations (ERA-20C, ERA-20C(bc)
respectively) and weather stations in Victoria (WS). \label{fig: Precipitation probability density functions for Victoria, Australia during the period 1974-2010 (r)}}
\end{figure}


\begin{figure}[h]
\noindent \begin{centering}
\includegraphics[scale=0.75]{Chapter_6/Figures/Fire_weather_variability/FFDI_variables/P/Time_series/1974-2010/P}
\par\end{centering}

\caption[Precipitation time series for the season December-January-February
(DJF) in Victoria during the period 1974-2010]{Precipitation time series for the season December-January-February
(DJF) in Victoria during the period 1974-2010. The correlation coefficients
between weather station and reanalysis data are r=0.49, p=0.00, n=37.
Data source: Original and bias-corrected European Reanalysis of Global
Climate Observations (ERA-20C, ERA-20C(bc) respectively) and weather
stations in Victoria (WS). \label{fig:Precipitation time series for the season December-January-February in Victoria during the period 1974-2010}}


\end{figure}



\subsubsection{McArthur Forest Fire Danger computation (1920-2010)}

The next step in this process was extending the FFDI calculation to
cover the period 1920-2010. Figure \ref{fig:Cumulative Forest Fire Danger Index time series for the season December-January-February in Victoria, Australia during the period 1920-2010 (r)}
shows the time series of cumulative FFDI. Panel a) displays the results
using the original reanalysis data (ERA-20C). On the other hand, panel
b) shows the results with bias-correction (ERA-20C(bc)). In both cases,
reanalysis time series are compared to the results using observations
(WS). Additionally, the analysis correlated the FFDI time series computed
with reanalysis and weather stations data (r=0.53, p=0.00). The bias-correction
decreases the magnitude of the FFDI variability to values to closer
to observations.

\begin{figure}[h]
\noindent \begin{centering}
\includegraphics[scale=0.75]{Chapter_6/Figures/Fire_weather_variability/FFDI/1920-2010/Time_series/FFDI_cum}
\par\end{centering}

\caption[Cumulative Forest Fire Danger Index (FFDI) time series for the season
December-January-February (DJF) in Victoria, Australia during the
period 1920-2010]{Cumulative Forest Fire Danger Index (FFDI) time series for the season
December-January-February (DJF) in Victoria, Australia during the
period 1920-2010. The correlation coeficients between weather station
and reanalysis data for the period 1974-2010 are a) r=0.51, p=0.00,
n=91 and b) r=0.53, p=0.00, n=91. Data source: Original and bias-corrected
European Reanalysis of Global Climate Observations (ERA-20C, ERA-20C(bc)
respectively) and weather stations in Victoria (WS). \label{fig:Cumulative Forest Fire Danger Index time series for the season December-January-February in Victoria, Australia during the period 1920-2010 (r)}}
\end{figure}


Figure \ref{fig:Number of days with Forest Fire Danger Index greater than 25 time series for the season December-January-February in Victoria, Australia during the period 1920-2010 (r)}
show time series of fire weather variability using the FFDI\textsubscript{>25}
metric. Panel a) and b) show the results with original and bias-corrected
reanalysis data. The bias-correction process adjusted the magnitude
of the FFDI values closer to observations. The relationship between
FFDI\textsubscript{>25} computed using weather station and reanalysis
is moderate (r=0.53, p=0.00, n=91).

\begin{figure}[h]
\noindent \begin{centering}
\includegraphics[scale=0.65]{Chapter_6/Figures/Fire_weather_variability/FFDI/1920-2010/Time_series/FFDI_g25}
\par\end{centering}

\caption[Number of days with Forest Fire Danger Index (FFDI) greater than 25
(FFDI\protect\textsubscript{>25}) time series for the season December-January-February (DJF)
in Victoria, Australia during the period 1920-2010]{Number of days with Forest Fire Danger Index (FFDI) greater than 25
(FFDI>25) time series for the season December-January-February (DJF)
in Victoria, Australia during the period 1920-2010. The correlation
coeficients between weather station and reanalysis data for the period
1974-2010 are a) r=0.53 p=0.00, n=91 and b) r=0.53 p=0.00, n=91. Data
source: Original and bias-corrected European Reanalysis of Global
Climate Observations (ERA-20C, ERA-20C(bc) respectively) and weather
stations in Victoria (WS). \label{fig:Number of days with Forest Fire Danger Index greater than 25 time series for the season December-January-February in Victoria, Australia during the period 1920-2010 (r)}}
\end{figure}


Subsequently, the investigation examined the seasonal variability
of each FFDI variable during the period 1920-2010. Figure \ref{fig:Forest Fire Danger Index variables time series during December-January-February in Victoria, Australia for the period 1920-2010 (r)}
shows four panels with their seasonal time series. Panel d), includes
the result of bias-correcting precipitation data. Panel a) and c)
show how temperature and wind speed do not display trends. In contrast,
panel b) and d) show an increasing trend for relative humidity and
precipitation.

\begin{figure}[h]
\noindent \begin{centering}
\includegraphics[scale=0.6]{Chapter_6/Figures/Fire_weather_variability/FFDI_variables/1920-2010-time-series/FFDI_variables}
\par\end{centering}

\caption[Forest Fire Danger Index (FFDI) variables time series during December-January-February
(DJF) in Victoria, Australia for the period 1920-2010]{Forest Fire Danger Index (FFDI) variables time series during December-January-February
(DJF) in Victoria, Australia for the period 1920-2010. Panels a),
b), c) and d) display time series for maximum temperature, relative
humidity, wind speed and precipitation respectively. Data source:
European Reanalysis of Global Climate Observations (ERA-20C). ERA-20C(bc)
corresponds to bias-corrected data. \label{fig:Forest Fire Danger Index variables time series during December-January-February in Victoria, Australia for the period 1920-2010 (r)}}
\end{figure}



\subsubsection{Victorian Seasonal Bushfire Index computation}

The analysis also included the computation of the VSBI using ERA-20C
data. The result using this data shows no fire weather trend for the
period 1920-2010 (see Figure \ref{fig:Victorian Seasonal Bushfire Index time series during December-January-February in Victoria, Australia for the period 1920-2010 (VSBI)}).
Additionally, Figure \ref{fig:Victorian Seasonal Bushfire Index variables time series during December-January-February in Victoria, Australia for the period 1920-2010 (r)}
displays time series for each of the components of the VSBI during
the same period. The figure shows four panels with time series of
temperature (panel a), relative humidity (panel b), sea surface temperature
(panel c), and precipitation (panel d). 

\clearpage

\begin{figure}[h]
\noindent \begin{centering}
\includegraphics{Chapter_6/Figures/Fire_weather_variability/VSBI/VSBI_ERA20C}
\par\end{centering}

\caption[Victorian Seasonal Bushfire Index time series during December-January-February
(DJF) in Victoria, Australia for the period 1920-2010]{Victorian Seasonal Bushfire Index time series during December-January-February
(DJF) in Victoria, Australia for the period 1920-2010. The computation
used the European Reanalysis of Global Climate Observations (ERA-20C).
The correlation coeficients between weather station and reanalysis
data for the period 1974-2010 are r=0.55 p=0.00, n=91 \label{fig:Victorian Seasonal Bushfire Index time series during December-January-February in Victoria, Australia for the period 1920-2010 (VSBI)} }
\end{figure}


\begin{figure}[h]
\noindent \begin{centering}
\includegraphics[scale=0.7]{Chapter_6/Figures/Fire_weather_variability/VSBI/VSBI_variables_1920_2010}
\par\end{centering}

\caption[Victorian Seasonal Bushfire Index variables for the period 1920-2010]{Victorian Seasonal Bushfire Index variables for the period 1920-2010.
The time series correspond to normalized spatial averages of a) temperature
(T), b) relative humidity (RH), and c) precipitation (P) over Victoria,
Australia during December-January-Februaary (DJF). These variables
correspond to the European Reanalysis of Global Climate Observations
(ERA-20C) data. Panel d) shows sea surface temperature (SST). This
time series uses data in the ``NINO1+2'' region from the HadISST
dataset. \label{fig:Victorian Seasonal Bushfire Index variables time series during December-January-February in Victoria, Australia for the period 1920-2010 (r)}}


\end{figure}



\subsection{Seasonal forecasting skill}

This section reproduces the seasonal forecasting skill analysis performed
in Chapter 5. However, this test uses ERA-20C data. The results show
the strength of the climate-bushfire relationships in Victoria. The
analysis included the antecedent (SON) and concurrent (DJF) bushfire
seasons. It also include the seasonal forecasting skill from spring
to summer. The figures display the correlations for the periods 1973-2009,
1960-2009, and 1919-2009 respectively. Figure \ref{fig:Pearson correlation coeficients for climate-bushfire relationships for September-October-November in Victoria, Australia during the period 1973-2009 (r)}
compares the results using weather station and reanalysis data. Coloured
circles represent statistically significant results a the 5\% level.


\subsubsection{Spring climate-bushfire relationships (SON-SON)}

Figures \ref{fig:Pearson correlation coefficients for climate-bushfire relationships for September-October-November in Victoria, Australia during the period 1973-2009 (r)},
\ref{fig:Pearson correlation coefficients for climate-bushfire relationships for September-October-November in Victoria, Australia during the period 1960-2009 (r)},
and \ref{fig:Pearson correlation coefficients for climate-bushfire relationships for September-October-November in Victoria, Australia during the period 1919-2009 (r)}
show the linear correlation coefficients for climate-bushfire relationships
in spring. The relationship between VSBI and FFDI metrics in spring
is strong. Using observations (see Figure \ref{fig:Pearson correlation coefficients for climate-bushfire relationships for September-October-November in Victoria, Australia during the period 1973-2009 (r)},
panel b), the correlations between VSBI and FFDI metrics are 0.92
and 0.78\textemdash with FFDI\textsubscript{cum} and FFDI\textsubscript{>25}
respectively\textemdash . Reanalysis data show a similar correlation
between VSBI and FFDI\textsubscript{cum}(r=0.90). This strong correlation
is virtually constant during the three periods analysed. In contrast,
the relationship between VSBI and FFDI\textsubscript{>25} is weaker
using reanalysis data. The relationship between these two metrics
ranges from 0.52 to 0.39.

Relative humidity is the climate variable that has the strongest influence
over fire weather. The correlations between relative humidity and
FFDI metrics are -0.97 and -0.81 with FFDI\textsubscript{cum} and
FFDI\textsubscript{>25} respectively\textemdash . The correlation
between relative humidity and FFDI\textsubscript{cum} using reanalysis
data is constant during every period. On the other hand, the strength
of the relationship between relative humidity and FFDI\textsubscript{>25}
decreases wih reanalysis data (-0.5<r<-0.37). 

Climate drivers have a moderate influence on fire weather in spring.
Observations show that the Indian Ocean Dipole has the strongest influence
over fire weather in this season (see Figure \ref{fig:Pearson correlation coeficients for climate-bushfire relationships for September-October-November in Victoria, Australia during the period 1973-2009 (r)},
panel b). Correlations between the Dipole Mode Index (DMI) and FFDI
metrics are r=0.5 and r=0.47\textemdash with FFDI\textsubscript{cum}
and FFDI\textsubscript{>25} respectively\textemdash . Among the ENSO
indicators, NINO3.4 shows the strongest correlations with fire weather\textemdash .
The correlations using weather station data show correlations of 0.43
and 0.38 between NINO3.4 with FFDI\textsubscript{cum} and FFDI\textsubscript{>25}
respectively. On the other hand, reanalysis data shows that NINO3.4
index has a slightly higher relationship with fire weather metrics
than the DMI. Using reanalysis data, the correlations between NINO3.4
and FFDI\textsubscript{cum  }range from 0.44 to 0.48; while the strength
of the relationship between DMI and FFDI\textsubscript{cum }ranges
from 0.42 to 0.44. ERA-20C is not able to capture statistically significant
relationship between climate indices and FFDI\textsubscript{>25}. 

\begin{figure}[h]
\noindent \begin{centering}
\includegraphics[scale=0.75]{Chapter_6/Figures/Fire_weather_variability/Correlations/1974-2010/ERA20CWS/SON-SON/SON-SON}
\par\end{centering}

\caption[Pearson correlation coefficients for climate-bushfire relationships
for September-October-November (SON) in Victoria, Australia during
the period 1973-2009]{Pearson correlation coefficients for climate-bushfire relationships
for September-October-November (SON) in Victoria, Australia during
the period 1973-2009. Data source: European Reanalysis of Global Climate
Observations (ERA-20C), Hadley Centre Sea Ice and Sea Surface Temperature
data set (HadISST), and weather station data in Victoria (WS). \label{fig:Pearson correlation coefficients for climate-bushfire relationships for September-October-November in Victoria, Australia during the period 1973-2009 (r)}}
\end{figure}


\begin{figure}[h]
\noindent \begin{centering}
\includegraphics[scale=0.75]{Chapter_6/Figures/Fire_weather_variability/Correlations/1961-2010/ERA-20C/SON-SON/SON-SON}
\par\end{centering}

\caption[Pearson correlation coefficients for climate-bushfire relationships
for September-October-November (SON) in Victoria, Australia during
the period 1960-2009]{Pearson correlation coefficients for climate-bushfire relationships
for September-October-November (SON) in Victoria, Australia during
the period 1960-2009. Data source: European Reanalysis of Global Climate
Observations (ERA-20C) and Hadley Centre Sea Ice and Sea Surface Temperature
data set (HadISST). \label{fig:Pearson correlation coefficients for climate-bushfire relationships for September-October-November in Victoria, Australia during the period 1960-2009 (r)}}
\end{figure}


\begin{figure}[h]
\noindent \begin{centering}
\includegraphics[scale=0.75]{Chapter_6/Figures/Fire_weather_variability/Correlations/1920-2010/ERA-20C/SON-SON/SON-SON}
\par\end{centering}

\caption[Pearson correlation coefficients for climate-bushfire relationships
for September-October-November (SON) in Victoria, Australia during
the period 1919-2009]{Pearson correlation coefficients for climate-bushfire relationships
for September-October-November (SON) in Victoria, Australia during
the period 1919-2009. Data source: European Reanalysis of Global Climate
Observations (ERA-20C) and Hadley Centre Sea Ice and Sea Surface Temperature
data set (HadISST). \label{fig:Pearson correlation coefficients for climate-bushfire relationships for September-October-November in Victoria, Australia during the period 1919-2009 (r)}}
\end{figure}



\subsubsection{Summer climate-bushfire relationships (DJF-DJF)}

Figures \ref{fig:Pearson correlation coefficients for climate-bushfire relationships for December-January-February in Victoria, Australia during the period 1974-2010 (r)},
\ref{fig:Pearson correlation coefficients for climate-bushfire relationships for December-January-February in Victoria, Australia during the period 1961-2010 (r)}
and \ref{fig:Pearson correlation coefficients for climate-bushfire relationships for December-January-February in Victoria, Australia during the period 1920-2010 (r)}
display summer climate-bushfire relationships. Overall, the graphs
display weaker\textemdash and less statistically significant\textemdash relationships
in comparison to spring. For example, correlations using weather station
data (see panel b in figure \ref{fig:Pearson correlation coefficients for climate-bushfire relationships for December-January-February in Victoria, Australia during the period 1974-2010 (r)})
show that the VSBI has a correlation of 0.69 and 0.65 with FFDI\textsubscript{cum}
and FFDI\textsubscript{>25 }respectively.

On the other hand, the correlations\textemdash using observations\textemdash between
the VSBI and fire activity are r=0.63 and r=0.5 with RF and DELWP
databases respectively (see Figure \ref{fig:Pearson correlation coefficients for climate-bushfire relationships for December-January-February in Victoria, Australia during the period 1974-2010 (r)},
panel b). These results are weaker than the correlations with fire
weather metrics. This relationship is an expected outcome since fire
activity no only depends on weather. Overall, using reanalysis data,
the correlations overestimate the relationship between VSBI and FFDI\textsubscript{cum}
which (0.89<r<0.91). In contrast, ERA-20C underestimates the association
between VSBI and fire activity records ( 0.35<r<0.53).

Observations show that relative humidity is the main fire driver in
this season. Additionally, observations demonstrate that relative
humidity has a stronger influence over fire weather than bushfire
events (see Figure\ref{fig:Pearson correlation coefficients for climate-bushfire relationships for December-January-February in Victoria, Australia during the period 1974-2010 (r)},
panel b). Reanalysis data agree to show this expected pattern. 

Remote climate drivers show a weak influence over fire metrics in
summer. Overall, the relationships are weaker using reanalysis data.
Additionally, the correlations decay using periods with more data.
During this season the IOD influence disappears (as expected). Therefore,
ENSO becomes the most important remote driver. 

\begin{figure}[h]
\noindent \begin{centering}
\includegraphics[scale=0.75]{Chapter_6/Figures/Fire_weather_variability/Correlations/1974-2010/ERA20CWS/DJF-DJF/DJF-DJF}
\par\end{centering}

\caption[Pearson correlation coeficients for climate-bushfire relationships
for December-January-February (DJF) in Victoria, Australia during
the period 1974-2010]{Pearson correlation coeficients for climate-bushfire relationships
for December-January-February (DJF) in Victoria, Australia during
the period 1974-2010. Data source: European Reanalysis of Global Climate
Observations (ERA-20C), Hadley Centre Sea Ice and Sea Surface Temperature
data set (HadISST), and weather stations data in Victoria. \label{fig:Pearson correlation coefficients for climate-bushfire relationships for December-January-February in Victoria, Australia during the period 1974-2010 (r)}}
\end{figure}


\begin{figure}[h]
\noindent \begin{centering}
\includegraphics[scale=0.75]{Chapter_6/Figures/Fire_weather_variability/Correlations/1961-2010/ERA-20C/DJF-DJF/DJF-DJF}
\par\end{centering}

\caption[Pearson correlation coefficients for climate-bushfire relationships
for December-January-February (DJF) in Victoria, Australia during
the period 1961-2010]{Pearson correlation coefficients for climate-bushfire relationships
for December-January-February (DJF) in Victoria, Australia during
the period 1961-2010. Data source: European Reanalysis of Global Climate
Observations (ERA-20C) and Hadley Centre Sea Ice and Sea Surface Temperature
data set (HadISST). \label{fig:Pearson correlation coefficients for climate-bushfire relationships for December-January-February in Victoria, Australia during the period 1961-2010 (r)}}
\end{figure}


\begin{figure}[h]
\noindent \begin{centering}
\includegraphics[scale=0.85]{Chapter_6/Figures/Fire_weather_variability/Correlations/1920-2010/ERA-20C/DJF-DJF/DJF-DJF}
\par\end{centering}

\caption[Pearson correlation coefficients for climate-bushfire relationships
for December-January-February (DJF) in Victoria, Australia during
the period 1920-2010]{Pearson correlation coefficients for climate-bushfire relationships
for December-January-February (DJF) in Victoria, Australia during
the period 1920-2010. Data source: European Reanalysis of Global Climate
Observations (ERA-20C) and Hadley Centre Sea Ice and Sea Surface Temperature
data set (HadISST). \label{fig:Pearson correlation coefficients for climate-bushfire relationships for December-January-February in Victoria, Australia during the period 1920-2010 (r)}}
\end{figure}



\subsubsection{Spring-Summer bushfire forecasting skill in Victoria (SON-DJF)}

Figures \ref{fig:Pearson correlation coefficients for climate-bushfire relationships from September-October-November to December-January-February in Victoria, Australia during the period 1973-2010 (part 1) (r)},
\ref{fig:Pearson correlation coefficients for climate-bushfire relationships from September-October-November to December-January-February in Victoria, Australia during the period 1973-2010 (part 2) (r)},
\ref{fig:Pearson correlation coefficients for climate-bushfire relationships from September-October-November to December-January-February in Victoria, Australia during the period 1960-2010}
and \ref{fig:Pearson correlation coefficients for climate-bushfire relationships from September-October-November to December-January-February in Victoria, Australia during the period 1919-2010 (r)}
show the climate-bushfire relationships from spring to summer. The
figures compare two types of correlations. The first group of correlations
show the relationships between climate variables, fire weather and
bushfire events. The second group includes climate indices instead
of climate variables in the correlations.

The results using observations (see Figure \ref{fig:Pearson correlation coefficients for climate-bushfire relationships from September-October-November to December-January-February in Victoria, Australia during the period 1973-2010 (part 1) (r)},
panel b) shows that the VSBI has a strong skill to forecast extreme
fire weather\textemdash r=0.74 and r=0.71 with FFDI\textsubscript{cum}
and FFDI\textsubscript{>25} respectively\textemdash . The VSBI correlations
also display a moderate skill to forecast fire activity\textendash r=0.66
and r=0.54 with RF and DELWP databases respectively\textemdash . The
predictive skill of this index is stronger than climate variables
and remote climate drivers alone. Moreover, using ERA-20C data the
VSBI skill to forecasting is still considerable. However, it also
shows inconsistencies. In most cases, using this data the skill is
higher to forecast fire activity than fire weather.

Correlations using observations and reanalysis data agree to show
that relative humidity is the climate variable with the strongest
prediction skill. The correlation matrix that uses observations (see
Figure \ref{fig:Pearson correlation coefficients for climate-bushfire relationships from September-October-November to December-January-February in Victoria, Australia during the period 1973-2010 (part 1) (r)},
panel b) illustrates that this variable has a higher forecasting skill
to predict fire weather than bushfires (as expected). The correlations
between relative humidity and FFDI metrics are r=-0.71 and -0.69\textemdash with
FFDI\textsubscript{cum} and FFDI\textsubscript{>25} respectively\textemdash .
Additionally, the correlations between this variable and fire records
are r=-0.64 and r=-0.58\textemdash with RF and DELWP respectively\textemdash However,
reanalysis data show that the forecasting skill of this variable is
higher for fire activity and weather (an inconsistency also detected
using the 20CR data). 

ENSO indicators are better fire predictors than the Dipole Mode Index
(DMI). Observation and reanalysis data agree to show this skill. Figure
\ref{fig:Pearson correlation coefficients for climate-bushfire relationships from September-October-November to December-January-February in Victoria, Australia during the period 1973-2010 (part 2) (r)}
(Panel b) shows the correlation matrix of climate indices, fire weather
and bushfires using weather station data. The correlations in this
figure are moderate. However, it indicates that ENSO indicators (based
on SST) show more statistically significant relationships with fire
metrics than the DMI. Among the ENSO indicators, NINO3.4 displays
higher correlations.

\begin{figure}[h]
\noindent \begin{centering}
\includegraphics[scale=0.75]{Chapter_6/Figures/Fire_weather_variability/Correlations/1974-2010/ERA20CWS/SON-DJF/SON-DJF_part_1}
\par\end{centering}

\caption[Pearson correlation coefficients for climate-bushfire relationships
from September-October-November (SON) to December-January-February
(DJF) in Victoria, Australia during the period 1973-2010]{Pearson correlation coefficients for climate-bushfire relationships
from September-October-November (SON) to December-January-February
(DJF) in Victoria, Australia during the period 1973-2010. Panel a)
and b) show the results for climate variables, fire weather and activity
using reanalysis and weather station data respectively. Data source:
European Reanalysis of Global Climate Observations (ERA-20C), Hadley
Centre Sea Ice and Sea Surface Temperature data set (HadISST), weather
stations in Victoria (WS). \label{fig:Pearson correlation coefficients for climate-bushfire relationships from September-October-November to December-January-February in Victoria, Australia during the period 1973-2010 (part 1) (r)}}
\end{figure}


\begin{figure}[h]
\noindent \begin{centering}
\includegraphics[scale=0.75]{Chapter_6/Figures/Fire_weather_variability/Correlations/1974-2010/ERA20CWS/SON-DJF/SON-DJF_part_2}
\par\end{centering}

\caption[Pearson correlation coefficients for climate-bushfire relationships
from September-October-November (SON) to December-January-February
(DJF) in Victoria, Australia during the period 1973-2010]{Pearson correlation coefficients for climate-bushfire relationships
from September-October-November (SON) to December-January-February
(DJF) in Victoria, Australia during the period 1973-2010. Panel a)
and b) show the results for climate indices, fire weather and activity
using reanalysis and weather station data respectively. Data source:
European Reanalysis of Global Climate Observations (ERA-20C), Hadley
Centre Sea Ice and Sea Surface Temperature data set (HadISST), weather
stations in Victoria (WS). \label{fig:Pearson correlation coeficients for climate-bushfire relationships from September-October-November to December-January-February in Victoria, Australia during the period 1973-2010 (part 2) (r)}}
\end{figure}


\begin{figure}[h]
\noindent \begin{centering}
\includegraphics[scale=0.75]{Chapter_6/Figures/Fire_weather_variability/Correlations/1961-2010/ERA-20C/SON-DJF/SON-DJF}
\par\end{centering}

\caption[Pearson correlation coefficients for climate-bushfire relationships
from September-October-November (SON) to December-January-February
(DJF) in Victoria, Australia during the period 1960-2010]{Pearson correlation coefficients for climate-bushfire relationships
from September-October-November (SON) to December-January-February
(DJF) in Victoria, Australia during the period 1960-2010. Panel a)
and b) show the results for climate variables and climate indices
respectively. Data source: European Reanalysis of Global Climate Observations
(ERA-20C) and Hadley Centre Sea Ice and Sea Surface Temperature data
set (HadISST). \label{fig:Pearson correlation coeficients for climate-bushfire relationships from September-October-November to December-January-February in Victoria, Australia during the period 1960-2010}}
\end{figure}


\begin{figure}[h]
\noindent \begin{centering}
\includegraphics[scale=0.75]{Chapter_6/Figures/Fire_weather_variability/Correlations/1920-2010/ERA-20C/SON-DJF/SON-DJF}
\par\end{centering}

\caption[Pearson correlation coefficients for climate-bushfire relationships
from September-October-November (SON) to December-January-February
(DJF) in Victoria, Australia during the period 1919-2010]{Pearson correlation coefficients for climate-bushfire relationships
from September-October-November (SON) to December-January-February
(DJF) in Victoria, Australia during the period 1919-2010. Panel a)
and b) show the results for climate variables and climate indices
respectively. Data source: European Reanalysis of Global Climate Observations
(ERA-20C) and Hadley Centre Sea Ice and Sea Surface Temperature data
set (HadISST). \label{fig:Pearson correlation coefficients for climate-bushfire relationships from September-October-November to December-January-February in Victoria, Australia during the period 1919-2010 (r)}}
\end{figure}



\section{Discussion}


\subsection{Bushfire and heatwave weather patterns}

The replication of the analysis comparing bushfire and heatwave weather
patterns confirms the results found in Chapter 4. The results are
robust for daily as well as seasonal time scales. The linkages with
ENSO patterns also remain valid. 

The daily composites show the most consistent results. For air temperature,
using alternative bushfire and heatwave data reveals that the differences
in their weather patterns are consistent. Cold fronts remain the most
important fire weather feature. On the other hand, persistent heat
anomalies characterize heatwave events. Relative humidity patterns
confirm that heatwaves in Victoria present dry anomalies in this region
with wet anomlies in several other areas of the continent. During
bushfires in Victoria, Australia is predominantly dry. The composites
produced using alternative data are also capable of reproducing the
contrasting intensity of the wave patterns during bushfires and heatwaves.
These anomalies also confirm the intense wave train pattern linked
to frontal activity during fire events. 

On a daily time scale, the results show that the intensity of the
frontal activity is the most important feature of extreme fire weather
in Victoria. Therefore, evaluating the frontal activity strength should
be a priority in daily fire weather forecasts. However, it seems that
the current state of practise heavily relies on the FFDI. This index
is most sensitive to temperature. Consequently, fire weather danger
estimated by the FFDI might be misleading during heatwave events. 

Alternative indices based on the effect of frontal activity over extreme
fire weather have already been published in the literature. \citet{Mills2005}
proposed a simple alternative based on this criteria. The author argues
that measuring the 850 hPa horizontal temperature gradient is a useful
metric to anticipate extreme fire weather. \citet{Reeder2015} proposed
a similar parameter. They used the difference in surface temperature
after cold front passages (\ensuremath{\Delta} 17 \textsuperscript{o}C)
to identify extreme fire weather days. In fact, \citet{Mills2005}
and \citet{Reeder2015} criteria have been used to study past extreme
fire weather and make projections (e.g. \citet{Hasson2009}). Since
there are already alternative fire weather metrics on a daily time
scale this research focused on the development of a new index for
seasonal prediction. Therefore, the seasonal patterns are also discussed.

On a seasonal time scale, bushfire and heatwave weather patterns also
display contrasts. During extreme fire activity in Victoria the weather
patterns consistently show warming anomalies in the tropical Pacific
Ocean. On the other hand, hetwaves do not seem to be linked to any
activity on this region. Perhaps, the most interesting pattern for
heatwaves was a dipole structure of warm and sligthly cool sea surface
temperature anomalies in the southern Indian Ocean. This suggests
that different mechanisms drive these two type of events on this time
scale. 

There is a link between \textquotedblleft El Ni\~no\textquotedblright{}
events and fire activity in Victoria. The linear correlation analyses
in this thesis, as well as in other studies (e.g. \citet{Harris2013}),
agree with this finding. However, this is the first study to use composites
to investigate this relationship. This methodology is useful to understand
the average condition of the climate system. Nevertheless, it also
has limitations. Not every \textquotedblleft El Ni\~no\textquotedblright{}
phase brings extreme fires. In fact, ENSO and fire activity in Victoria
have a moderate\textemdash or even weak\textemdash inter-annual relationship.
Analysing the weather pattern recurrence in different ENSO phases
could provide further insights. For example, using this approach \citet{Verdon-Kidd2008a}
investigated synoptic patterns useful to forecast rainfall in this
region. 

An additional limitation to investigate weather patterns is the lack
of long-term fire records. The seasonal composites are sensitive to
this caveat. For example, scarce areas in the tropical Pacific Ocean
show statistical significance. The areas selected to develop the VSBI
are an exception. This result contrasts with the strong statistical
significance that the daily composites display. This absence of statistical
significance occurs with the two bushfire databases used. The \textquotedblleft El
Ni\~no\textquotedblright{} signal is clear during fire season. However,
the analysis requires more data to have greater confidence in the
results. 

In contrast to fire weather, heatwaves do not show a strong association
with ENSO. The results are consistent using two different heatwave
definitions. This finding agrees with \citet{Boschat2014}. The authors
showed that most heatwaves in Victoria occur in neutral ENSO conditions\textemdash .
In contrast, \citet{Parker2014} found that Victoria tends to experience
heatwaves events during \textquotedblleft La Ni\~na\textquotedblright{}
phase. Evidently, these disagreements require further investigation.

Future investigation about seasonal heatwave weather patterns could
consider several aspects. First of all, extending the period of analysis
is feasible. There is available data from the early years of the 20\textsuperscript{th}
century to undertake this analysis. However, this study aimed to compare
heatwave and bushfire weather patterns. Therefore the period of analysis
comprises the year with consistent fire records (1961-2010). Additionally,
future investigations could focus on investigating the intra-seasonal
relationships that enhance heatwaves. Low-frequency modes of variability
do not seem to drive heatwaves. Finally, simple composite analysis
limit the ability to discuss the dynamics with climate drivers. Therefore,
complementing the composites with numerical experiments could provide
further insights (e.g. \citet{Sadler2012}). 


\subsection{Fire weather variability \label{sub:Seasonal-fire-weather-variability}}

This investigation shows an increasing fire danger trend in Victoria
during the period 1974-2010. This finding is based on observations
and agrees with similar studies \citep{Lucas2007,Lucas2010,Clarke2013}.
Not surprisingly, FFDI metrics display a greater increase in fire
danger than the VSBI. This difference occurs because the FFDI is more
sensitive to temperature. This variable displays clear increasing
trends. Additionally, this tendency is stronger considering every
day in the fire season. The FFDI\textsubscript{cum} metric shows
a steeper increase in fire danger than FFDI\textsubscript{>25}. However,
FFDI\textsubscript{>25} might not accurately identify the extreme
fire weather days. As already discussed, cold fronts characterise
extreme fire weather days better than extreme temperature. This section
also discusses the relationship between ENSO and fire weather variability. 

Strong \textquotedblleft El Ni\~no\textquotedblright{} years bring high
fire danger to Victoria. The VSBI shows that the years with the highest
fire danger are those during a strong \textquotedblleft El Ni\~no\textquotedblright{}
phase (e.g. 1982-1983 and 1997-1998). However, the VSBI might overestimate
ENSO's influence on fire danger. The FFDI shows that strong \textquotedblleft El
Ni\~no\textquotedblright{} years are not the most threatening. Although,
this index demonstrates that they can pose a significant risk. FFDI
metrics show that the seasons 1982-1983 and 1997-1998 are among the
most dangerous during the period 1974-2010. The FFDI\textsubscript{cum}
metric categorise them among the five seasons with the highest fire
danger. Using FFDI\textsubscript{>25}, these seasons are in the top
seven years. The discussion about ENSO and fire weather metrics continues
in the next section. This section also aims to discuss the fire weather
changes before the observational period.

Reanalysis data shows substantial differences in their ability to
represent long-term fire weather changes. Overall, ERA-20C has a greater
skill to capture fire weather variability than 20CR. Indeed, this
reanalysis product shows better correlations with observations for
two reasons. First of all, it has a higher resolution (ERA-20C: 1.5\textsuperscript{o}x1.5\textsuperscript{o},
20CR: 2\textsuperscript{o}x2\textsuperscript{o}). Additionally,
ERA-20C yields surface (10 m) climate fields. On the other hand, the
20CR (version 2) provides data only at the 1000 hPa level. Therefore,
most fire weather variables did not require bias-corrections using
ERA-20C. On the other hand, the 20CR better represents the observed
fire danger trend in Victoria. The 20CR data also shows this trend
for the period 1974-2010. In fact, it shows that this tendecy can
be traced back to the early decades of the 20\textsuperscript{th}
century. Additionally, this trend is consistent using three fire weather
metrics (FFDI\textsubscript{cum}, FFDI\textsubscript{>25} and VSBI).
However, the results using ERA-20C do not support this finding.

Understanding the renalysis disagreements requires discussing the
behaviour of fire weather variables individually. Australia has long-term
observations for temperature \citep{Trewin2013} and rainfall \citep{Laverly1997}.
Using these records, \citet{Ashcroft2014} analysed climate variability
in southeastern Australia for the period 1860-2009. This research
also used the 20CR data. Therefore, it is useful for verification.
The study shows that temperature observations have a clear increasing
trend. On the other hand, observed rainfall displays the opposite
pattern. The 20CR agrees with these long-term trends. Although, there
is greater uncertainty before 1900. In contrast, this thesis shows
that ERA-20C is not able to capture the observed trends. Further reasearch
could quantify uncertainties in past fire weather changes using different
ensembles of the 20CR. 


\subsection{Fire weather metrics, drivers and predictability}

This section re-examines several aspects already discussed in Chapter
5. First of all, the performance of the VSBI in comparison to seasonal
metrics of the FFDI. Additionally, the influence of local climate
variables and remote drivers on fire weather. Finally, it discusses
the potential to predict extreme fire weather in Victoria.

The VSBI shows that it can represent fire weather and its relationship
with bushfires. For example, the index demonstrated to be an equivalent
metric in comparison to the FFDI. The results show that they are highly
correlated during spring and summer. Moreover, VSBI and FFDI display
similar correlations with fire activity. The index also showed to
be a useful forecasting tool (as discussed later in this section). 

In addition to its accuracy, the VSBI has the advantage to be simpler
to compute than the FFDI. It represents fire danger based on a simple
relationship between climate variables and ENSO. Therefore, it could
become useful in studies that require large computing resources. In
contrast, the FFDI incorporate factors and sub-indices that increase
the computation complexity.

On the other hand, the VSBI also has caveats. First of all, it does
not represent the ``true'' physical relationship between its variables.
In fact, the VSBI linear formulation masks their real contribution
to fire danger. For example, the SST influence is lower in comparison
to local climate variables. Therefore, the VSBI equation might overestimate
the SST effect on fire weather. The linear formulation of the VSBI
also fails to reflect the actual non-linear nature of ENSO. Additionally,
some of the VSBI variables are directly correlated (e.g. temperature
and relative humidity). 

Despite its limitations, the VSBI was useful to understand fundamental
climate-bushfire relationships. The correlations showed that regional
climate conditions drive fire weather in Victoria. In contrast, remote
drivers have less influence than climate variables (as expected).
These findings agree with \citet{Harris2013}. However, climate modes
of variability do play a role in driving fire weather. For example,
ENSO and IOD moderately influence relative humidity. This variable
showed to be the principal component driving fire weather. Additionally,
these remote drivers also demonstrated to have a moderate influence
on fire activity. However, ENSO and IOD's influence is not constant. 

The influence of remote climate drivers over fire weather decays from
spring to summer. During spring, the IOD exerts a slightly higher
level of control than ENSO (according to observations). Indeed, this
study found no strong evidence to show that the IOD has a significantly
higher influence than ENSO during this season. In fact, these two
climate modes are highly correlated. The correlations displayed in
Chapter 5 show similar results. These findings contrast with \citet{Cai2009}
who argue that the IOD is the leading driver that preconditions fuel
loads to fires. On the other hand, during summer, the IOD does not
exert influence over fire weather. This behaviour was expected from
its seasonal dynamics. Unexpectedly, the influence of ENSO weakens
during this season. Therefore, the effect of remote climate drivers
is greater in spring than summer. This finding supports the argument
that a fire weather forecasting tool can be developed using \textemdash antecedent\textemdash remote
climate drivers information. 

The VSBI shows a strong forecasting skill. Indeed, the index has greater
predictive power than climate variables and indices on their own.
Additionally, the correlations show a greater skill to predict fire
weather than bushfires. This result shows physical consistency. These
findings are based on the analysis of observations. However, some
of the disagreements found in Chapter 5 using the 20CR persisted using
in the analysis with ERA-20C. For example, a greater predictive ability
for forecast fire events rather than fire weather. 

Reanalysis datasets showed similar limitations to reproduce the results
using observations. Particularly, in the last two chapters. In contrast,
they showed agreement when used to investigate large-scale weather
patterns. Therefore, their resolution seems to be a caveat to conduct
this regional climate study. Probably, better results could be achieved
increasing the size of the study region (e.g. southeastern Australia).


\section{Summary}

This chapter aimed to reproduce the results generated for Victoria.
The use of alternative data confirmed the differences between bushfire
and heatwave weather patterns in this region. A strong temperature
gradient in southern Australia\textemdash produced by cold fronts\textemdash and
a dry pattern over the continent characterise extreme fire weather.
In contrast, heatwaves show persistent hot anomalies and a contrasting
dry-wet pattern in Australia. 

Additionally, the use of an alternative reanalysis dataset shows that
the models have different capabilities to represent fire weather.
The 20CR dataset captures the observed\textemdash increasing\textemdash trend
in fire weather. On the other hand, ERA-20C showed better skill to
represent inter-annual fire weather changes. The ERA-20C data also
verified that the VSBI has a similar skill than FFDI metrics to represent
fire weather. 

Furthermore, a re-examination of the results using observations confirms
that the VSBI has a strong fire weather forecasting skill. The incorporation
of an ENSO indicator in this index shows that it increases its predictive
capability. 

The next chapter investigates fire weather in Ecuador. The study is
probably the first to explore this topic in the tropical Andes. It
aims to verify if the McArthur Forest Fire Danger Index is a useful
metric in this region. The chapter also investigates the relationship
between ENSO and fire weather. This research provides insights of
ENSO's influence in modulating fire weather in the Southern Hemisphere. 

