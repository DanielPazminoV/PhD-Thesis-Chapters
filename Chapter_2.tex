
\chapter{Background} 
\newpage{}


\section{Introduction}

The introduction chapter explained the motivation, aims and scope
of this research project. It also provided a general context for the
study. This chapter describes some of the most important concepts
used throughout the thesis. Explaining them is important because it
provides background for the next chapters, in particular for the literature
review. Additionally, Chapter 2 expands on the context setting with
a characterization of the study regions: Victoria, Australia and the
Ecuadorian Andes. 

Four concepts are essential for this thesis. These are fire weather,
heatwaves, climate modes of variability and seasonal forecasting.
Fire weather is the main topic of this thesis. This chapter examines
its main variables and its quantification in Australia. It also provides
a brief description of heatwaves. The chapter explains the difficulties
in defining these events and its similarities with fire weather. Additionally,
some of the drivers of fire weather are large-scale phenomena known
as climate modes of variability. The background describes three climate
modes relevant for this project. These are the El Ni\~no-Southern Oscillation,
the Indian Ocean Dipole and the Southern Annular Mode. Chapter 2 also
provides a brief introduction to seasonal weather forecasting techniques. 

On the other hand, the characterization of the study regions focuses
on three aspects. The geography, climate and fire history of Victoria
and Ecuador. The geography characterization describes the significant
landscape features of both areas. The climate description explains
the major drivers, classification of climatic regions and climatologies.
The fire section describes the causes and impacts of bushfires in
these regions.


\section{Definition of concepts}


\subsection{Fire weather}

The primary focus of this investigation is fire weather. Therefore,
this section will introduce some important ideas to understand this
topic. The fire weather background comprises three aspects. First,
it explains which meteorological variables contribute to severe fire conditions.
Additionally, it explains how they exert their influence. This section
also describes how mathematical models represent the effect of fire
weather variables on bushfires. It includes a description of the McArthur
Forest Fire Danger Index (FFDI) to explain the quantification of fire
danger in Australia. Finally, the relationship between fire weather
and heatwaves is also described. 


\subsubsection{Fire weather variables \label{sub:Fire-weather-variables}}

Fire weather is analysed through variables such as air temperature,
relative humidity, wind, and precipitation. The behaviour of these
meteorological variables influences every aspect of bushfires, including
their ignition, rate of spread and intensity. Bushfires start if high temperatures,
low levels of relative humidity and precipitation have dried vegetation
to a flammable point. They also require an ignition source, natural or human-related.
Winds also contribute in the severity of bushfires
by taking moisture content out of the vegetation and providing oxygen
for combustion \citep{Bradstock2012}. The spread of bushfires is
also controlled by winds, which propagate them and also can change
their flank \citep{BureauofMeteorology2008}. The intensity of bushfires
is determined by the flammability and distribution of vegetation;
which are controlled by precipitation on seasonal time scales \citep{Bradstock2012}. 

The effect of fire weather variables on bushfires has been described
in several mathematical models. These models represent the ignition,
spread or intensity of bushfires. Modelling efforts focus on one of
these aspects or a combination of them. For example, the Forest Fire
Danger Index (FFDI) \citep{McArthur1966,McArthur1967} quantifies
fire danger focusing on the effect of weather on vegetation. 
Therefore,this index targets ignition and rate of spread, as well as ease of control. 
On the other hand, the Haines Index basis
is the effect of atmospheric instability in producing spotting \citep{Haines}.
This index focuses on spread and intensity. The Canadian Forest Fire
Weather System \citep{VanWagner1974} models the three aspects. In
most cases, fire weather modelling efforts require a calibration process.
The calibrations account for local conditions such as the type of
vegetation and local climate and weather. This investigation uses the FFDI to explore
fire weather in Victoria and Ecuador because of its calibration criteria. 


\subsubsection{Forest Fire Danger Index (FFDI)}

Australian fire authorities use the FFDI for bushfire risk management
\citep{Luke1978}. The index was originally formulated by Alan Grant
McArthur \citep{McArthur1966,McArthur1967}. Initially, operators
used mechanical meters to calculate the index \citep{Luke1978}. A
set of carboard wheels comprised these devices. The meters facilitated
the introduction or ``dialling'' of weather data to calculate the
index. The FFDI was later formulated as equations by \citet{Noble1980}.
The equation that represents the index is:

\begin{equation}
FFDI=2e^{(-0.45+0.987ln(DF)-0.0345RH+0.0338T+0.0234v)}
\end{equation}


Where:

$T$: Air temperature (\textsuperscript{o}C).

$v$: Wind speed (Km/h).

$RH$: Relative humidity (\%).

$DF$: Drought factor.

The computation of the index requires using values at a specific time of day when the maximum FFDI is likely to occur.
The index values usually peak in the afternoon at 3 pm. This is the standard
time to compute the FFDI in Australia \citep{Lucas2010}. The index
also requires the computation of a drought factor. 


\paragraph{Drought factor:}

This factor represents the effect of cumulative precipitation on the dryness of fuel loads.
\citet{Griffiths1999} developed the Drought Factor (DF) equation
and calculation procedures. This contribution corresponds to an improved
formula originally proposed by \citet{Noble1980}. The DF ranges from
1 to 10. These values depend on the number of days since the last ``significant''
rain day. The identification of this day requires the analysis of
the last 20 consecutive days. The sum of the whole range is assigned
to the day of the highest rain. This value accounts for the ``long-term''
effect of precipitation. Appendix A includes further details regarding
the computation of the drought factor.

The DF formula, originally expressed as \textquotedblleft D\textquotedblright{}
by \citet{Griffiths1999} is:

\begin{equation}
D=10.5\left(1-\exp^{-\left(I+30/40\right)}\left(\frac{\left(41y^{2}+y\right)}{\left(40y^{2}+y+1\right)}\right)\right)
\end{equation}


Where:

$D$: Drought Factor. 

$I$: Soil dryness index (mm equivalent). 

The significant rain is any event that maximizes $y$, according to
the following cases shown in table \ref{tab:Significant-rain-event}

\begin{table}[h]
\caption[Significant rain event cases for drought factor calculation]{Significant rain event cases. Adapted from \citet{Griffiths1999}
\label{tab:Significant-rain-event}}


\noindent \centering{}%
\begin{tabular}{|c|c|c|}
\hline 
Case number & Significant rain factor & Conditions\tabularnewline
\hline 
\hline 
1 & $y=\frac{N^{1.3}}{\left(N^{1.3}+P-2\right)}$ & If N $\geq$1 and P > 2\tabularnewline
\hline 
2 & $y$= $\frac{0.8^{1.3}}{\left(0.8^{1.3}+P-2\right)}$ & If N=0 and P >2\tabularnewline
\hline 
3 & y = 1 & If P $\leq$ 2\tabularnewline
\hline 
\end{tabular}
\end{table}


Where: 

$N$: Time since the rain event in days. 

$P$: Rainfall in mm during an event. 

The $DF$ also requires the computation of a soil dryness index. Soil
dryness represents the amount of rain necessary to saturate the upper
200 mm of the soil layer. This step represents the short-term effect
of precipitation on fuels. Typical soil dryness indices to compute
the DF are those proposed by \citet{KeetchJJ} or \citet{Mount1972}.
The computation of the $DF$ in Australia commonly uses the \citet{KeetchJJ}
method.


\paragraph{Keetch-Byram Drought Index (KBDI):}

\citet{KeetchJJ} developed the index for the United States Department
of Agriculture-Forest Service. The index used an eight inches layer
of soil for evaluation, yielding a scale that ranges from 0 to 800.
A zero value represents saturation. Daily rainfall and maximum temperature
are fundamental variables for the calculation of this index. Additionally,
its calculation is only possible for areas where the mean annual rainfall
is known.


\paragraph{Danger classes:}

The FFDI has six danger classes. Originally, the index was calibrated
for a maximum value of 100, with FFDI of 100 occurring in an extreme fire event occurred in 1939. Over the last years, weather conditions
in catastrophic bushfire events have yielded values above this limit
\citep{Lucas2010}. Figure \ref{fig:Forest Fire Danger Index danger classes}
presents the danger classes used in fire management operations:

\begin{figure}[h]
\noindent \begin{centering}
\includegraphics[scale=0.8]{Chapter_2/Figures/Bushfire-Risk-Rating-Chart}
\par\end{centering}

\caption[Forest Fire Danger Index danger classes]{Forest Fire Danger Index danger classes. Source: Deparment of Land
Resource Management of the Australian Nothern Territory Government.
\label{fig:Forest Fire Danger Index danger classes} }
\end{figure}



\subsection{Heatwaves}

There is no universal definition for heatwaves \citep{Nairn2013,Perkins2012}.
However, most definitions characterize them as consecutive days with
abnormally high temperatures \citep{Nairn2013}. Selecting ``hot''
days usually requires the adoption of absolute thresholds or percentiles thresholds.  \citep{Perkins2012}.
Using an arbitrary number of consecutive days above a fixed temperature
value\textemdash also arbitrary\textemdash is common \citep{Nairn2013,Perkins2012}.
These subjective decisions make comparisons among heatwave definitions
difficult \citep{Perkins2012}. For example, in Adelaide, the declaration
of a heatwave requires five consecutive days with maximum temperature
exceeding 35 \textsuperscript{o}C \citep{Nairn2013}. In the same
city, a heatwave is also declared after three consecutive days above
40 \textsuperscript{o}C. On the other hand, setting thresholds that
include minimum temperature is also common \citep{Perkins2012,Pezza2012}.
This variable accounts for the dissipation of heat at night after
a hot day. 

Temperature is an important variable to define heatwaves as well as
fire weather. For this reason, heatwaves are usually linked with increased
bushfire risk \citep{Nairn2013,Perkins2012,Pezza2012}. Therefore,
it is important to understand the processes that link\textemdash or
distinguish\textemdash these two types of events. 

Heatwaves in southeast Australia occur because of three mechanisms \citep{McBride2009}.
First, the advection of hot and dry air. An anticyclone advects air
from the north desserts to the forested areas in southern Australia.
Secondly, the subsidence of air in an anticyclone brings atmospheric stability.
This process produces clear skies that increase solar radiation. Finally,
the increased insolation generates increased surface heat fluxes. These fluxes
heat the air in the mixed layer. Fire weather is associated with these processes
as well, although, its analysis involves more variables and interactions
(see subsection \ref{sub:Victoria-fire}). 


\subsection{Climate modes of variability}

Large-scale meteorological and oceanographic phenomena drive natural
climate variability. These drivers are known as modes of climate variability.
For example, El Ni\~no-Southern Oscillation (ENSO), Indian Ocean Dipole
(IOD), and Southern Annular Mode (SAM) are important modes of climate variability in the Southern Hemisphere \citep{Bjerknes1966,Bjerknes1969,Aceituno1988,Saji1999,ThompsonandSolomon2002}.
This section describes ENSO and IOD since they can exert significant
influence on fire weather \citep{Williams1999,Cai2009}. Despite its
importance driving precipitation variations in Victoria, the influence of SAM
over fire weather has not been investigated. 


\subsubsection{El Ni\~no-Southern Oscillation (ENSO)}

ENSO is a large-scale process that drives climate on a global scale
\citep{Bjerknes1966,Bjerknes1969}. During ENSO events, the tropical
Pacifc Ocean interacts with the atmosphere in a coupled system \citep{Bjerknes1969}.
The most common manifestation of this process is the warming and cooling
of sea surface temperature (SST). Warm conditions on the east side
of the tropical Pacific Ocean represent ``El Ni\~no'' conditions.
This state usually brings above-average rainfall in the coasts of
Ecuador and Peru \citep{Aceituno1988}, while bringing hot and dry
conditions in Indonesia and Australia \citep{Williams1999,Wooster2012}.
On the other hand, cool anomalies in the same region represent ``La
Ni\~na'' conditions. The ENSO phenomena is referred as oscillation
since these two phases\textemdash warm and cool\textemdash have periods of approximately three to seven 
years \citep{NOAA}. 

The normal state of the tropical Pacific Ocean is relatively cool
waters in the east and warm conditions to the west. Convection usually
takes place in the west, which generates easterly winds near to the
surface of the ocean. The hot and moist air mass originated in the
west travel with a westerly direction in the upper troposphere. Along
the way it loses heat and moisture. Cool and dry air sinks in the
east side of the ocean reinforcing the circulation pattern. This circulation
pattern is known as ``Walker'' circulation, a name coined by \citet{Bjerknes1969}
in honour to Sir Gilbert Thomas Walker. 

A weakening of the easterlies reduces the upwelling in the eastern
part of the tropical Pacific Ocean. This produces an increase in the
sea surface temperature, which transfers heat to the atmosphere weakening
the Walker circulation and in extreme cases reversing it. These changes
in sea surface temperature produced by the changes in wind stress
over the ocean produce ``El Ni\~no'' events. This change in the atmospheric
circulation pattern can have effects in places distant from where it started
(e.g. floods and droughts). This type of response is referred to as a ``teleconnection''. 

At the same time, the changes in the circulation of air masses produced
by the heat flux\textemdash changes in sea surface temperatures\textemdash produce
variations in surface pressures along the tropical Pacific. This is
the main manifestation of a larger scale phenomena called the ``Southern
Oscillation'' \citep{Bjerknes1969}. For this reason this phenomena
is regarded as coupled ocean-atmophere process. The name ``El Ni\~no-Southern
Oscillation'' aims to reflect this complexity. 

A common ENSO metric is the Southern Oscillation Index (SOI). The
SOI represents the difference in normalized surface pressure between
Taihiti and Darwin, Australia \citep{NOAA}. This research adopted
the SOI\textquoteright s calculation method proposed by the United
States National Oceanic and Atmospheric Administration \citep{NOAA}.
The index reflects the changes in air-mass in the tropical
Pacific Ocean. These fluxes, are a manifestation of changes in SST.
The SOI has an open ended scale with positive and negative values.
A negative SOI value represents above-average warm SST conditions
in the eastern tropical Pacific Ocean. This represents an ``El Ni\~no''
phase of the phenomenon. The opposite pattern occurs with a positive
SOI value, representing a ``La Ni\~na'' phase.

Another ENSO index is the Oceanic Ni\~no Index (ONI) \citep{NOAAa}.
The ONI measures monthly SST values in the tropical Pacific Ocean.
The measurements take place in the region from 5 \textsuperscript{o}N
to 5 \textsuperscript{o}S latitude and 120 \textsuperscript{o}W
to 170 \textsuperscript{o}W longitude. This is the NINO3.4 (a.k.a
``Ni\~no 3.4'') region. The index requires the calculation of 3-month
running means and its climatology. The ONI is the difference between
these two values. An ``El Ni\~no'' condition exits if there is at
least an anomaly of +0.5 \textsuperscript{o}C in this computation.
An anomaly of -0.5 \textsuperscript{o}C indicates a ``La Ni\~na''
phase. Figure \ref{fig:El Ni=0000F1o-Southern Oscillation regions}
shows the El Ni\~no-Southern Oscillation regions in the tropical Pacific
Ocean.

\begin{figure}[h]
\noindent \begin{centering}
\includegraphics[scale=0.66]{Chapter_2/Figures/El_nino_regions}
\par\end{centering}

\caption[El Ni\~no-Southern Oscillation regions]{El Ni\~no-Southern Oscillation regions. The boxes delimit the regions
used to compute the Oceanic Ni\~no Index (ONI). Source: United States
National Oceanic and Atmospheric Administration (NOAA) \label{fig:El Ni=0000F1o-Southern Oscillation regions}}
\end{figure}



\subsubsection{Indian Ocean Dipole}

The Indian Ocean Dipole (IOD) is another coupled ocean-atmosphere
physical process \citep{Saji1999}. The IOD is characterized by cool
sea surface temperatures in Sumatra and warmer than average SST in
the western side of the Indian Ocean basin\textemdash positive IOD\textemdash (see
Figure \ref{fig:Positive phase of the Indian Ocean Dipole}). During
a negative IOD process the dipole structure is the opposite. The IOD
explains 12\% of the variance in sea suface temperature in this region
\citep{Saji1999}. \citet{Saji1999} argues that this is a process
independent of ENSO, although this is a matter of ongoing debate \citep{Dommenget2002}. 
In fact, most positive IOD events in September-November are associated with "El Ni\~no" events.
 

\begin{figure}[h]
\noindent \begin{centering}
\includegraphics[scale=0.4]{Chapter_2/Figures/Indian_Ocean_Dipole}
\par\end{centering}

\caption[Positive phase of the Indian Ocean Dipole]{Positive phase of the Indian Ocean Dipole. Warm sea surface temperature
on the western side of the Indian Ocean basin that contrast with cool
conditions in Sumatra characterise these events. Source: Japan Agency
for Marine-Earth Science and Technology (JAMSTEC) . \label{fig:Positive phase of the Indian Ocean Dipole}}
\end{figure}


However, the mechanisms that drive IOD and ENSO events are similar.
A positive IOD process starts with the cooling of SST in Sumatra.
A wekenning\textemdash or even reversal\textemdash of the westerly
winds close to sea surface enhances the SST warming in the west side
of the Indian Ocean basin. The warm SST produce convection, which
drives precipitation over eastern Africa. The hot and moist air produced
by the convection process travel on a westerly direction. On the upper
troposphere, the air mass dries and cools, and then sinks over Sumatra.
The subsidence of these air masses produce drought conditions over
Indonesia. The entire process starts in May and peaks in October \citep{Saji1999}.
Posite IOD events bring dry spring conditions to Victoria \citep{Cai2009}. 

The intensity of the Indian Ocean Dipole is usually measured with
the ``Dipole Mode Index'' (DMI) \citep{Saji1999}. The DMI is the
difference of averaged SST anomalies in the west (50 \textsuperscript{o}E-70
\textsuperscript{o}E, 10 \textsuperscript{o}S-10 \textsuperscript{o}N)
and east (90 \textsuperscript{o}E-110 \textsuperscript{o}E, 10 \textsuperscript{o}S-Equator)
Indian Ocean \citep{Saji1999}. This simple index is useful in representing the variance of SST in this region. 


\subsubsection{Southern Annular Mode}

The Southern Annular Mode (SAM) is the main source of monthly
climate variability in the extra-tropical Southern Hemisphere \citep{ThompsonandSolomon2002}.
It refers to variations in the strength of the the zonally symmetric circulation around Antartica. These winds surround Antartica
as a belt. The SAM produces an expansion or contraction of the zonal bands of wind. 

This phenomena is caused by the non-uniform heating of the Earth \citep{KingandTurner2007}.
This heating regulates heat and mass transport that creates zones
of high and low pressure in the planet. The intensity of the pressure
gradient between midlatitudes and latitudes close to the South Pole
is a measure of the strength of SAM. Anomalously high pressures in midlatitudes and low
pressures close to the poles strenghten the circumpolar vortex. The
strengthenning process shifts the circulation slightly closer to the pole. This
contraction of the circulation belt produces below-average rainfall
in many regions of the Southern Hemisphere \citep{Ho2011}. This process
refers to a positive SAM event (see Figure \ref{fig:Positive phase of the Southern Annular Mode}).
On the other hand, when the pressure anomalies show the opposite pattern
the circumpolar vortex is weakened and the circulation is shifted
away from the South Pole. The effect on precipitation over the same
regions is the opposite. This produces a negative SAM event.

\begin{figure}[h]
\noindent \begin{centering}
\includegraphics[scale=0.5]{Chapter_2/Figures/Southern_Annular_Mode}
\par\end{centering}

\caption[Positive phase of the Southern Annular Mode]{Positive phase of the Southern Annular Mode. Higher pressures in midlatitudes
and low pressures in Antartica bring dry conditions to Victoria, Australia.
Source: \citet{Jones2012}. \label{fig:Positive phase of the Southern Annular Mode}}
\end{figure}


There are many indices to measure the intensity of this mode of variability.
\citet{Gong1999} defined SAM as the difference of normalized monthly
pressure between 40 \textsuperscript{o}S and 65 \textsuperscript{o}S.
Another commonly used index is the Antartic Annular Oscillation Index
(AAO) \citep{NOAA_2016_AAO}. The AAO is the leading mode of an Empirical
Orthogonal Function analysis using monthly mean 700 hPa geopotential height anomalies. 


\subsection{Seasonal weather forecasting}

Climate modes of variability have a strong influence on local weather.
This influence can be exerted on seasonal time scales. Investigating
seasonal fire weather forecasting using climate indices is an active
area of research \citep{Kitzberger2002,Nicholls2007,Harris2013}.
This section describes some approaches used in seasonal weather forecasting.
It also briefly describes seasonal fire weather forecasting limitations
and feasibility in Australia.


\subsubsection{Statistical and dynamical forecasting}

Seasonal forecasting uses statistical-empirical methods as well as
dynamical models \citep{Doblas-Reyes2013,Troccoli2005}. Statistical-empirical
models describe the relationship between predictands and predictors
usually assuming linear relationships \citep{Troccoli2005}. Common
predictands are temperature and precipitation \citep{Doblas-Reyes2013,Pepler2015,Troccoli2005}.
Usual predictors are sea surface temperature, soil moisture, snow
cover and sea ice \citep{Doblas-Reyes2013,Troccoli2005}. One of the
most common problems with this approach is multicollinearity \citep{Troccoli2005}.
This occurs when there are predictors in the forecast model that are
correlated.

On the other hand, dynamical forecasts rely on the physical modelling
of the climate system. A simple model describes the interactions of
the atmosphere only. In complex models, all the components of the
system and their interactions take part. Setting initial conditions
and insufficient knowledge of physical processes are common problems
with them \citep{Doblas-Reyes2013,Troccoli2005}.

Dynamical models have recently started to have better forecasting accuracy than
statistical-empirical methods \citep{Doblas-Reyes2013}. Additionally,
ensemble dynamical models perform better than a single model \citep{Pepler2015}.
Yet, sophisticated statistical methods can have an accuracy comparable
to the dynamical counterparts \citep{Doblas-Reyes2013}. In practise,
decision-makers use a combination of statistical-empirical and dynamical
approaches for seasonal forecasts \citep{Doblas-Reyes2013}. 


\subsubsection{Seasonal Fire weather forecasting}

Seasonal fire weather forecasting use the predictions of variables such as
temperature and precipitation. Their forecast is better for mean values
than for extremes \citep{Doblas-Reyes2013,Pepler2015}. However, it
also requires information about relative humidity and wind. The skill
to forecast this variables is less developed \citep{Roads2005}. Therefore,
the study of fire weather prediction has also explored other techniques.
For example, the investigation of seasonal prediction associated with
large-scale events like ENSO. 

In Australia, there are regions that exhibit a relationship between
fire activity and ENSO. Tasmania exhibits more area burnt with decreased
rainfall in the coincident summer season \citep{Nicholls2007}. Since
ENSO is a major driver of Australian rainfall \citep{Risbey2009b};
\citet{Nicholls2007} suggested its indices can contribute to forecast
fire activity in this region. Additionally, the Northern Territory
exhibits a strong relationship between area burnt, rainfall and ENSO
\citep{Harris2008}. In Victoria, there is also potential to use ENSO
indices for fire activity forecasting \citep{Harris2013}. However,
\citet{Nicholls2007} acknowledged that a link between ENSO and fire
activity may not exist in other regions of the country.


\section{Characterization of the study regions}

This investigation explores fire weather in two regions of the Southern
Hemisphere: Victoria, Australia and the Ecuadorian Andes. These regions
were selected because they share the same bushfire-prone vegetation:
predominantly \textit{Eucalyptus} forests. They also share ENSO as a strong climate
driver. This section characterizes these regions's geography, climate
and fire history.


\subsection{Geography}

Victoria and the Ecuadorian Andes display different landscapes. The
most remarkable difference is the elevation of the mountain ranges
that divide their territory. This characterization describes data
on location, area, and landscape features. 


\subsubsection{Victoria}

Victoria is located in southeastern Australia ranging in latitude
from 34\textsuperscript{o} 2\textquoteright{} S to 39\textsuperscript{o}
8\textquoteright{} S (see Fig. \ref{fig:Localization of the state of Victoria, Australia}).
This state occupies 227,618 Km\textsuperscript{2} \citep{Sherbon1975}
and has a population of 5.9 million people \citep{ABS}. The Great
Dividing Range is its most important physical feature \citep{Luke1978}.
These mountain ranges comprise the Victorian Alps at the north-east
of the state. The maximum elevation in Victoria reaches 1,986 m at
Mount Bogong. The elevation difference creates several climatic regions
in Victoria. The Murray River Basin plains comprise large areas of
agricultural land that extend to the west of the Great Dividing Range.
In contrast, the north-west of Victoria or ``Mallee'' region is
a flat and desert area. Victoria's southern limit is the Bass Strait.
The southern lowlands of Victoria encompasses Melbourne, its capital
city and major urban centre with a population of 4,4 million people
\citep{ABS}. 

On the other hand, most rural areas of Victoria comprise agricultural and large forested
areas. Many of native forest areas occur in national parks. For example, Grampians
National Park is located to the west of Melbourne. It has an extension
of 167,219 Ha \citep{StateGovernmentVictoria2016}. The Wilsons Promontory
National Park is located to the south-east of the city. The park comprises
an area of 50,500 Ha \citep{StateGovernmentVictoria2016}. To the
east of the city, Gippsland is a region that also has extensive forested
areas. The vegetation of Victoria makes it prone to bushfires \citep{Williams2012}.
The \textit{Eucalyptus} (Myrtaceae) is the dominant specie in these
forests \citep{Williams2012}. Non Eucalyptus forests include the
specie \textit{Melaleuca} (Myraceae) and the \textit{Callitris} (Cupressaceae)
\citep{Williams2012}. 

\begin{figure}[H]
\noindent \centering{}\includegraphics{Chapter_2/Figures/Victoria}\caption{Location of the state of Victoria, Australia.\label{fig:Localization of the state of Victoria, Australia}}
\end{figure}



\subsubsection{Ecuador}

Ecuador is located in the north-west side of South-America (see Figure
\ref{fig:Location of Ecuador}). The country has a diverse geography.
It comprises a continental territory between the latitudes 01\textsuperscript{o}
28\textquoteright{} N and 05\textsuperscript{o} 02\textquoteright{}
S and longitudes 75\textsuperscript{o} 11\textquoteright{} W and
81\textsuperscript{o} 04\textquoteright{} W \citep{InstitutoGeograficoMilitarIGM2013a}. The national territory
has a total area of 256,370 Km\textsuperscript{2}. In the year 2010,
the country had a population of 14,5 million people \citep{InstitutoGeograficoMilitarIGM2013a}.
The Andes mountains are the most important geographical feature of
Ecuador \citep{Insel2010}. They divide the continental Ecuador into
three areas: the Coast, Andean and Amazon regions. Each of these regions
has diverse geographical features, climate, and ecosystems. 

\begin{figure}[h]
\noindent \begin{centering}
\includegraphics[scale=0.25]{Chapter_2/Figures/Ecuador_location}
\par\end{centering}

\caption[Location of Ecuador]{Location of Ecuador. \label{fig:Location of Ecuador}}
\end{figure}


Most bushfires in Ecuador occur in the Andean region. The Ecuadorian
Andes comprises two flanks of mountains: the eastern and western ``Cordilleras''.
Between these two flanks\textemdash that include active volcanoes\textemdash there
are several inter-Andean valleys. Many of these Andean plateaus are
over 3,000 m above sea level. The steep elevations have an important
effect on local climate. The atmospheric lapse rate create the conditions
for different bio-climatic zones to exit \citep{Pourrut1995}. Yet,
the original vegetation of the Andean valleys has almost disappeared
\citep{Anchaluisa2013}. Most of the endemic vegetation was replaced
by the Australian specie \textit{Eucalyptus globulus} around 1860
\citep{MinisteriodelAmbiente2012}. Other introduced species are the
\textit{Pinus radiata} (from California) and the \textit{Pinus patula}
(from Mexico). The introduction of these species occurred for commercial
reasons \citep{Anchaluisa2013}, for managed forests. These species are prone to bushfires
in the Andean dry season. 


\subsection{Climate and weather}

\LyXZeroWidthSpace This section provides general information about
climate and weather in the two study regions. This characterization
includes the description of their main climate drivers, seasonality and regionalization
of its climate and climatologies. This is important because Victoria
and the Ecuadorian Andes have especific climate characteristics that
influence fire weather. 


\subsubsection{Climate in Victoria}

ENSO, IOD, and SAM drive Victoria's inter-annual climate variability
\citep[and references therein]{Verdon-Kidd2008}. These drivers can
produce wetter or drier conditions in this region. This influence
depends on the phase of each climate mode. For example, strong ``El
Ni\~no'' events can bring hot and dry conditions to Victoria \citep{BoM_ENSO_2016}.
The opposite is true for ``La Ni\~na'' conditions. Additionally, positive
SAM and IOD phases reduce rainfall in Victoria \citep{Cai2009,Ho2011}.
Their negative phases have the opposite effect effect. 

On the other hand, the influence of these climate modes also varies
between seasons. For instance, ENSO is important during the entire
year \citep{BoM_ENSO_2016}. SAM reduces rainfall in winter while
increasing it in spring and summer \citep{Hendon2007}. The IOD influences
temperature and precipitation in Victoria particularly in winter and
spring \citep{Cai2009,Risbey2009b}. However, the interactions between
them are not entirely understood \citep{Verdon-Kidd2008}. In fact,
\citet{Risbey2009b} argue that they all dependent on each other,
especially ENSO and the IOD. In spite of these complex interactions,
it is possible to regionalise Victoria's climate. 

Australia uses three methods to classify climate regions \citep{BoM}.
These classification schemes are based on temperature, vegetation
and seasonal rainfall criteria respectively. Since bushfires are linked
to the effect of climate on fuel loads, the classification based on
the vegetation criteria is perhaps the most appropiate to describe
Victoria's climate. This classification originally proposed by K�ppen
in early years of the 20\textsuperscript{th} century describes Victoria
as a temperate region (see figure \ref{fig:Climate classification of Australia}).
Overall, in a temperate climate summers are hot and dry; while winters
are cool and wet. Victoria's climate is also influenced by its topography
\citet{BureauofMeteorology2008}.

\begin{figure}[h]
\noindent \begin{centering}
\includegraphics[scale=0.5]{Chapter_2/Figures/Climate_classification}
\par\end{centering}

\caption[Climate classification of Australia based on a modified K�ppen classification
system]{Climate classification of Australia based on a modified K�ppen classification
system. Victoria is a temperate region. Source: Australian Bureau
of Meteorology. \label{fig:Climate classification of Australia}}
\end{figure}


The Great Dividing Range (GDR), has created three main climatic zones:
northern, southern and southeastern Victoria \citep{Luke1978}. In
northern Victoria, precipitation is influenced by northwesterly air
masses. The north-west region of this climatic zone receives an annual
average of 250 to 300 mm of precipitation and endures intense heatwaves;
whereas, in the north-east the presence of mountains raises the precipitation
range to 800-1000 mm in a year. Precipitation in southern Victoria
also increases with altitude; however, westerly winds drive its formation.
In contrast, variations in temperature in this zone are driven by
northerly winds, which produce summers with average air temperature
of 28 \textsuperscript{o}C and maxima of 43 \textsuperscript{o}C or more.
In southeastern Victoria, synoptic systems in the Tasman Sea influence
its precipitation rate, especially in the eastern region that recieves
on average 1000 mm a year. The western region of southeastern Victoria
is less influenced by the Tasman Sea, receiving on average 600 mm
of precipitation.


\subsubsection{Climate in Ecuador}

Since most bushfires in Ecuador occur in the highlands, the climate
characterization focuses on this region. The Amazon forest, oceanic
currents, and the topography are among the main climate drivers in
the tropical Andes \citep{Martinez2011}. The Amazon forests produce
a great amount of water vapor. Topography forces this water to precipitate
in the western Cordillera. Orographic precipitation also occurs in
the eastern Cordillera. Air masses advected from the Pacific Ocean
bring moisture to this flank of the Andes. The influence of the Pacific
Ocean in the tropical Andes climate variability is important. The
inter-annual variability on this region depends on ENSO \citep{Vuille2000,Villacis2003,Martinez2011}.
These influences create several climate regions within the Andes. 

Most of the Ecuadorian Andes\textemdash between 1500 to 3000 m\textemdash have
a \textquotedblleft semi-humid mesothermal\textquotedblright{} climate
\citep{Pourrut1995}. Mean temperatures in this region range from
8 \textsuperscript{o}C to 20 \textsuperscript{o}C. Maximum temperatures
span from 22 \textsuperscript{o}C and 30 \textsuperscript{o}C. Minimum
temperatures range from -4 \textsuperscript{o}C to 5 \textsuperscript{o}C.
Air masses from the Pacific Ocean and the Amazon region create three
seasons \citep{Pourrut1995}. Two wet seasons during the periods February-May
and October-November. A dry season spans from June to September. Total
annual precipitation ranges from 800 mm to 1500 mm along the Andes.
Relative humidity varies from 65\% to 85\%.

\begin{figure}[h]
\noindent \begin{centering}
\includegraphics[scale=0.5]{Chapter_2/Figures/climate_of_Ecuador}
\par\end{centering}

\caption[Climate classification of Ecuador]{Climate classification of Ecuador. A \textquotedblleft semi-humid
mesothermal\textquotedblright{} climate characterises the Ecuadorian
Andes. Adapted from \citep{Pourrut1995}. \label{fig:Climate classification of Ecuador}}
\end{figure}



\subsection{Fire}

Fire history is better documented in Victoria than Ecuador. Therefore,
the characterization in Victoria uses concrete examples of catastrophic
bushfire events that occurred in the last 100 years. More general information
describes the situation for Ecuador, with an emphasis on recent events.


\subsubsection{Victoria\label{sub:Victoria-fire}}

Several conditions make Victoria vulnerable to bushfires. First of
all, this state is the most densely populated and urbanised in Australia
\citep{Bryant2008}. Although it only covers 5\% of its territory
\citep{Luke1978}, 24.7\% of Australia's population lives here \citep{Bryant2008}.
Additionally, forest cover one-third of its territory \citep{Luke1978}.
This region also has an extended fire season. Victoria's fire season
comprises the austral summer season December-February \citep{Luke1978,BoM2009}
(see Figure \ref{fig:Fire seasons in Australia}). During these months
fire activity is more intense. Although, extreme bushfires can also
occur in November and March.

\begin{figure}[h]
\noindent \begin{centering}
\includegraphics[scale=0.75]{Chapter_2/Figures/Fire_seasons}
\par\end{centering}

\caption[Fire seasons in Australia]{Fire seasons in Australia. Fire seasons in Victoria occur in the austral
summer (December-January-February). Source: Australian Bureau of Meteorology.
\label{fig:Fire seasons in Australia}}
\end{figure}


Bushfires in Victoria have produced massive impacts. There is more
documentation of them after the European settlement \citep{DEPIa}.
For example, half of the economic loss in Australia due to bushfires occurred in this
region \citep{Luke1978}. Most of these impacts correspond to major
catastrophic events \citep{Blanchi2012}. Three of the most serious
catastrophes were \textquotedblleft Black Friday\textquotedblright (1939),
\textquotedblleft Ash Wednesday\textquotedblright (1983), and \textquotedblleft Black
Saturday\textquotedblright{} (2009) \citep{DEPIa}. To better understand
bushfires in Victoria it is convenient to examine these major events.

The \textquotedblleft Black Friday\textquotedblright{} bushfires occurred
on January 13\textsuperscript{th}, 1939 \citep{DEPIa}. In this event,
71 people died and fires burned 1.5 to 2 million hectares \citep{DEPIa}.
This event motivated authorities to create a special commission\textemdash Royal
Comission\textemdash to investigate its causes deeply. A prolonged
drought, and a hot and dry summer created the conditions for this
event to occur. Additionally, the causes of the fires were anthropogenic
\citep{DEPIa}. This commission fostered laws and institutions that
aim to prevent and control bushfires \citep{DEPIa}. Some of the institutions
exist even today. For example, the former Forests Commission, now
Department of Environment, Land, Water, and Planning.

\textquotedblleft Ash Wednesday\textquotedblright{} occurred on February
16\textsuperscript{th}, 1983 \citep{DEPIa}. This event generated
over 100 fires across the state \citep{DEPIa}. The fires killed 47
people and burned 210,000 hectares \citep{DEPIa}. Approximately 82,500
hectares were public land including areas in the Dandenong Ranges
National Park close to Melbourne\citep{DEPIa}. A 10-month drought preceded these fires
\citep{DEPIa}. On the day of the fires, temperatures reached 40 \textsuperscript{o}C,
and relative humidity was below 15\%. Hot and dry northerly winds
created these conditions. Additionally, a strong cold front increased
the difficulty to suppress the fires \citep{DEPIa}. This cold front
changed the direction of the fire flank and increased its size. The
causes of the fires were accidental, deliberate and some not identified
\citep{DEPIa}.

The last catastrophic event was the worst in Australian history. \textquotedblleft Black
Saturday\textquotedblright{} occurred on February 7\textsuperscript{th},
2009 \citep{Teague2010}. These fires killed 173 people \citep{Teague2010}
and destroyed 430,000 hectares \citep{DEPIa}. A prolonged heatwave
preceded the event \citep{Teague2010}. This heatwave produced three
consecutive days with temperature above 43 \textsuperscript{o}C in
Melbourne\textemdash capital of Victoria\textemdash{} \citep{VBRC_2009_final_report}. Other areas in Victoria
experienced record-breaking temperatures reaching 48 \textsuperscript{o}C
on \textquotedblleft Black Saturday\textquotedblright{} \citep{Teague2010}.
These fires motivated the establishment of another commission\textemdash 2009
Victorian Bushfires Royal Comission\textemdash . The commission clearly
identified the characteristics of the fires. The main features included:
rapid spread, fires crowned in forested areas, convection columns,
extensive spotting, and a cold front that changed the direction of
the fires and their extension \citep{Teague2010}. Figure \ref{fig: =00201CBlack Saturday=00201D fires}
illustrates the intensity of the fires. The commission estimated the
economic losses of these fires to reach \$4 billion \citep{Teague2010}.

\begin{figure}[h]
\noindent \begin{centering}
\includegraphics[scale=0.36]{Chapter_2/Figures/Black_saturday}
\par\end{centering}

\caption[``Black Saturday'' fires]{``Black Saturday'' fires. Fire columns during ``Black Saturday''.
Photo taken by Andrew Brownbill. \label{fig: =00201CBlack Saturday=00201D fires}}
\end{figure}



\subsubsection{Ecuador}

Bushfires occur in three geographical regions of Ecuador. However,
the Ecuadorian Andes endures most of the bushfire occurrences. This
region has large extensions of introduced \textit{Eucalyptus} forests
\citep{Anchaluisa2013}. Thus, the availability of fuels makes it
the most bushfire-prone region. The fire season in the Ecuadorian
Andes spans from July to November \citep{SecretariadeAmbiente2013}.
Yet, the most critical months are July, August and September \citep{Estacio2012,SecretariadeAmbiente2013}.
The coast and the Galapagos Islands experience bushfire episodes from
January to May \citep{SecretariadeAmbiente2013}.

Bushfires produce severe impacts in Ecuador. Yet, information about
these impacts is scarce and disperse. Bushfires produced 21,570 Ha
of area burnt in Ecuador in 2012 \citep{MinisteriodelAmbiente2013}.
The Metropolitan District of Quito (MDQ) is particularly vulnerable
to this hazard. The vulnerability is high because 15.4\% of the national
population lives here. Additionally, vegetation covers 60.46 \% of
its territory \citep{SecretariadeAmbiente2013}. Bushfires destroyed
2700 Ha of the MDQ vegetation in 2009 \citep{Estacio2012}. In 2012,
the area burnt by bushfires increased to 4,882.16 Ha \citep{SecretariadeAmbiente2013}.
This fire season cost the MDQ more than 50 million dollars \citep{SecretariadeAmbiente2013}.
In 2015, extreme bushfires occured again in Quito killing three firefighters
(see http://www.elcomercio.com/actualidad/quito-bombero-muerte-incendio-puembo.html).
Figure \ref{Bushfires in the Ecuadorian Andean region} depict the
magnitude of these fires.

\begin{figure}[h]
\noindent \begin{centering}
\includegraphics[scale=0.5]{Chapter_2/Figures/Incendios_forestales_quito}
\par\end{centering}

\caption[Bushfires in the Ecuadorian Andean region]{Bushfires in the Ecuadorian Andean region. Extreme bushfire in the
urban-rural interface of Quito\textemdash capital of Ecuador\textemdash{}
during the 2012 season. Source: http://www.panorama.com.ve \label{Bushfires in the Ecuadorian Andean region}}
\end{figure}


Unfortunately, bushfires in Ecuador are often anthropogenic in nature \citep{MinisteriodelAmbiente2013}.
In fact, the MDQ reported that humans cause 95 percent of bushfires
in the city \citep{SecretariadeAmbiente2013}. Burning forests to
convert them to agricultural use is the main cause of bushfires in
the country \citep{MinisteriodelAmbiente2013,SecretariadeAmbiente2013,Rodas2015}.
Forcing land use changes for urbanization purposes is another reason
people start bushfires \citep{SecretariadeAmbiente2013}. Rural communities
burn agricultural waste, which is a traditional practice that fosters
bushfire occurrence \citep{Estacio2012}. Rural people also burn
regular waste because of the lack of waste management services \citep{SecretariadeAmbiente2013}.
Finally, bushfires occur because of the damaging acts of arsonists
\citep{MinisteriodelAmbiente2013,SecretariadeAmbiente2013}.


\section{Summary}

This chapter comprised two parts. First, it explained important concepts
used in this thesis. Additionally, it also characterized the study
regions for this research: Victoria, Australia and the Ecuadorian
Andean region. This summary highlights the most important aspects
of both sections. 

Four concepts are essential for this thesis. The first concept is
fire weather. It refers to the meteorological conditions that foster
bushfire occurrence. The main fire weather variables are temperature,
relative humidity, wind speed and precipitation. These variables can
be combined in mathematical equations to evaluate fire danger. On
the other hand, it is difficult to define heatwaves. However, most
definitions use an arbitrary number of days with temperature above
a fixed threshold. The third concept is climate modes of variability.
These are large-scale meteorological and oceanographic phenomena.
They can influence local climate from remote places. The Southern
Hemisphere has three important climate modes of variability. These
are the El Ni\~no-Southern Oscillation (ENSO), the Indian Ocean Dipole
(IOD), and the Southern Annular Mode (SAM). Finally, a sub-section
briefly explained seasonal weather forecasting. It refers to the use
of weather variables to predict weather behaviour in lead times of
months. Statistical and dynamical models are used in seasonal weather
forecasting. This chapter also characterized the study regions of
this research. 

The characterization featured several links between Victoria and Ecuador.
First of all, vegetation is their landscape connection. They share
the \textit{Eucalyptus} as the dominant forest species in bushfire-prone
areas. The two regions also share ENSO as a climate driver. In fact,
ENSO is the main remote climate driver for Ecuador. However, Victoria's
climate is more complex. Interactions between ENSO, the IOD, and the
SAM strongly influence its climate. On the other hand, Victoria's
fire history is well documented. It has endured catastrophic bushfire
events over the last 100 years. Finally, Ecuador lacks comprehensive
fire history documentation. However, data obtained from recent years
suggests that bushfire impacts in Ecuador are severe. This hazard
endangers one of the most biodiverse countries in the world. 

The definition of essential concepts and the characterization of the
study regions was a necessary step. This information provides background
to analyse critically the relevant scientific literature. The next
chapter presents the literature review for this project.

