
\chapter{Literature review}
\newpage{}


\section{Introduction}

The background chapter provided information to understand the basic
science and research context. This chapter presents a more specific,
detailed and critical review of the scientific literature. After the
introduction, five sections comprise this literature review. The first
part analyses the synoptic patterns linked to fire weather. Subsequently,
the chapter discusses investigations of bushfire danger using the
McArthur Forest Fire Danger Index. The review also examines the use
of climate variables and ENSO indices in seasonal bushfire forecasting.
The synthesis of research gaps follows these discussions. Finally,
the last section summarizes the literature review. 

The review has a bias towards Australia since it has generated more
research than South America on these topics. The first two sections
do not include information on this region. However, fire weather forecasting
in South America has received some research attention. Therefore,
its studies are also included in this review.


\section{Fire weather synoptic patterns }

Synoptic patterns help us understand how the climate system drives
meteorological events \citep{Sturman2006}. For example, fire weather
in southeastern Australia has a typical pattern. Over the Tasman Sea,
a quasi-stationary anticyclone is present \citep{BoM2009}. Its circulation
advects hot and dry air from the interior of the continent \citep{Mills2005,Engel2013,Reeder2015}.
These environmental conditions make vegetation prone to fire ignitions
\citep{Williams2012}. At the same time, a cyclone over the Southern
Ocean supports the arrival of cold fronts to this region \citep{Reeder1987}.
These cold fronts can produce a catastrophic bushfire event because
they change and extend the flank of the fire fronts \citep{Mills2005,Engel2013,Reeder2015}.
Some of the investigations that describe this pattern focus on the
catastrophic events \citep{Mills2005,Mills2005a,Cruz2012,Engel2013,Fiddes2015,Reeder2015}.

\citet{Mills2005} performed an evaluation of the meteorological conditions
of ``Ash Wednesday''. The author proposed the 850 hPa level horizontal
temperature gradient south to Australia as a parameter to assess fire
weather. Extreme fires are more likely to occur when there are greater
temperature differences between the cold front air mass and the hot
column of air over southern Australia. Using this criterion, \citet{Mills2005}
established that \textquotedbl{}Ash Wednesday\textquotedbl{} ranks
in the 0.1\% of events using 40 years of reanalysis data. This temperature
gradient parameter could be useful for forecasting purposes and has
already been used to assess future extreme fire seasons \citep{Hasson2009}.
On the other hand, reproducing the arrival of cold fronts is another
key factor in forecasting extreme fire weather \citep{Mills2005,Engel2013}. 

\citet{Mills2005} investigation also comprised a simulation that
provided further insights on the wind structures that made \textquotedblleft Ash
Wednesday\textquotedblright{} catastrophic. It highlighted that not
only the mobility of the frontal system created this extreme fire
event, but also the depth of the trough system. In this study, the
cold front arrival was simulated with a 1-hour lag. Over time, computer
simulations of extreme fire events in this region have become more
sophisticated. 

\citet{Engel2013} simulated the meteorological conditions of ``Black
Saturday''. Their study used the UK Unified Model to simulate with
great precision the mesoscale phenomena of this event. The model was
able to capture horizontal convective rolls and nocturnal bores. It
was also capable of reproducing the arrival of the cold front of that
day with a 30 minutes lag compared to observations.

\citet{Mills2005} and \citet{Engel2013} are examples of a typical
approach to examine fire weather in this region. Several studies focus
on a single case study to simulate and explain fire weather behavior
\citep{Mills2005,Mills2005a,Cruz2012,Engel2013}. \textquotedblleft Ash
Wednesday\textquotedblright{} and ``Black Saturday'' are the most
relevant cases regarding human fatalities and economic impact \citep{Blanchi2010,Blanchi2012},
and therefore the focus of great research attention. Investigating
single case fire weather studies has been useful to identify the main
characteristics of these events. However, fire weather research with
a long-term synoptic approach has been less common. 

\citet{Reeder2015} provided further insights on fire weather patterns
with a climatological perspective. They investigated the role of Rossby
waves and extreme fronts on fire weather. The analysis covered the
period 1979-2010 and focused on southeastern Australia. The study
used days with intense frontal activity to define extreme fire weather
events. A surface temperature drop of a least of 17 \textsuperscript{o}C
( \textgreek{D}T > 17 \textsuperscript{o}C ) defined extreme cold
fronts. Identifying them proved to be useful in establishing fire
weather days. \citet{Reeder2015} argue that Rossby wave breaking
produces a potential vorticity anomaly in this region. According to
the author, this anomaly generates an anticyclone in the Tasman Sea
that drives hot and dry air to southeastern Australia. 

The circulation produced by an anticyclone in the Tasman Sea has also
been attributed to the occurrence of heatwaves \citep{Pezza2012,Parker2013,Boschat2014}.
In fact, a recent study investigated the oceanic influences that enhanced
the ``Black Saturday'' extreme heatwave and cold front \citep{Fiddes2015}.
However, \citet{Reeder2015} claim that fire weather and heatwave
conditions are usually different. On one hand, this study argues that
cold fronts are the defining feature of fire weather. On the other
hand, several studies mention that fire and heatwaves are related
events without providing evidence or further references \citep{Pezza2012,Parker2013,Boschat2014}.
Therefore, the especific characteristics that distinguish them and
their interactions remain unexplored. Investigating this relationship
demands to review some relevant heatwave literature.

\citet{Pezza2012} investigated the synoptic climatology of heatwaves
in southern Australia. In this study, three consecutive days with
maximum temperature above the 90\textsuperscript{th} percentile defined
a heatwave. The criteria included the lack of relief during the nights.
In their hetwave definition, the minimum temperature on the second
and third days should also be above the 90\textsuperscript{th} percentile.
The definition required using a monthly climatology. The period of
analysis was 1979-2008. The investigation used an automatic tracking
scheme \citep{Murray1999} to investigate the trajectories of the
cyclones and anticyclones. The study determined that an anticyclone
in the Tasman Sea produces heatwave conditions in southeastern Australia
advecting hot and dry air to the interior of the continent. Other
studies concur with this mechanism \citep{Parker2013,Parker2014a}.
According to \citet{Pezza2012}, a wave train amplification that starts
in the Indian Ocean creates the anticyclone. 

In contrast to fire weather research, the investigation of heatwaves
in this region has considered the interactions with other extreme
events. \citet{Boschat2014} examined the large-scale connections
between heatwaves in Victoria and extreme rainfall events in Queensland.
The co-occurrence of these events motivated this study. However, this
reasecrh found that there is no dynamic link between them. Tropical
convection drives extreme rainfall events in Queensland. This convection
usually occurs during strong \textquotedblleft La Ni\~na\textquotedblright{}
events. On the other hand, heatwaves in Victoria are locally driven
by the mechanisms described by \citet{Pezza2012}. In contrast, \citet{Parker2014a}
found that tropical cyclone outflow reinforces the potencial vorticity
of the antyclone that produces heatwave conditions in this region. 

\citet{Pezza2012}, \citet{Boschat2014} and \citet{Parker2014a}
investigations undertake the synoptic climatology approach that seems
less explored in fire weather research. It appears that a lack of
robust fire activity records hampers these type of studies. For example,
\citet{Mills2005} and \citet{Reeder2015} rely first on weather parameters
to identify extreme fire days. \citet{Mills2005} used the 850 hPa
horizontal temperature gradient before the cold front arrival. On
the other hand, \citet{Reeder2015} used the temperature difference
after the passage of the cold front. In both cases, they validated
they approaches with fire activity dates published in the scientific
literature. These dates are scarce. Therefore, the logic behind their
approaches is sound. Eventhough the official bushfire database of
the Victorian Government has not been peer-reviewed, it has been used
in other fire investigations \citep{Harris2013}. Therefore, it could
be useful to investigate long-term fire weather patterns. Additionally,
extending the synoptic climatology of heatwaves is another reasearch
task that could be undertaken.

The investigation of long-term heatwave weather patterns in southeastern
Australia is feasible. Maximum and minimum temperature are the basic
variables to compute heatwaves. Long-term and high-quality records
of these variables span the entire 20\textsuperscript{th} century
in Australia. For example, the Australian Climate Observations Reference
Network - Surface Air Temperature (ACORN-SAT) dataset compiled these
records \citep{Trewin2013}. Additionally, there are reanalysis products
that span the 20\textsuperscript{th} century \citep{Compo2011,Stickler2014}.
Despite this, most heatwaves studies on southeastern Australia start
their analyses only after the nineteen-seventies \citep{Pezza2012,Parker2013,Boschat2014,Parker2014a}.
This time scope hampers the ability to reach conclusions about trends. 


\section{Fire weather characterized by the McArthur Forest Fire Danger Index}

The previous section described synoptic patterns linked to fire activity.
Fire weather indices offer a complementary approach to fire research.
These indices quantify the risk that weather conditions pose to fire
occurrence and behaviour. They are also useful metrics to assess the
changes in fire danger over extended periods of time. The use of the
McArthur Forest Fire Danger Index (FFDI) has been common in Australia.
Therefore, this section reviews the literature that evaluated fire
weather variability using this index. An analysis of fire weather
projections complements this discussion.

The FFDI has been the basis to understand fire weather in Australia.
\citet{Williams1999} found that \textquotedblleft El Ni\~no\textquotedblright{}
events increase fire danger in most of central and eastern Australia.
\citet{Long2006} found that extreme fire weather in Victoria occurred
with a wind flowing from the north or north-west. \citet{Blanchi2010}
showed that the majority of house losses during bushfires in Australia
occured in the most extreme fire weather days. \citet{Lucas2010}
developed a historical fire weather database for Australia using weather
station data. \citet{Clarke2013a} demonstrated that the Weather Reasearch
and Forecasting (WRF) model \citep{Skamarock2008} was useful to simulate
the spatial distribution of fire danger in southeastern Australia.
\citet{Engel2013} examined the spatial and temporal variability of
fire danger during ``Black Saturday''. These studies used the FFDI
to represent fire weather in Australia. To my knowledge, the FFDI
has not been used internationally. However, \citet{Field2015} suggest
its use to contrast their results in the development of a global fire
weather dataset using the Canadian Fire Weather Index (FWI) \citep{VanWagner1974}.
The FFDI has also been used to analyse fire weather trends.

Fire weather in southeastern Australia has become more dangerous.
\citet{Lucas2007} investigated fire weather recent trends in southeast
Australia. They found that there were significant positive trends
in several metrics of the FFDI. \citet{Clarke2013} agree with these
results. In this study, the authors built on the work of \citet{Lucas2010}
calculating the FFDI in more stations, extending the period of analysis,
and homogenizing wind speed data for the computations. \citet{Clarke2013}
analysed trends with FFDI metrics and found that there are significant
increases in fire danger in Australia\textemdash especially in the
southeast of the continent\textemdash . These studies describe fire
weather variability after the nineteen-seventies. However, extending
the FFDI record would provide valuable information about past fire
weather changes. This type of research remains a challenge due to
the lack of weather stations records. Using reanalysis products could
be an alternative approach to address this problem. In contrast, the
investigation of future fire weather has received more research attention.

Climate projections suggest that fire danger will continue its increasing
trend in southeastern Australia. \citet{Lucas2007} determined that
the number of days with extreme fire weather might experience an increase
in the range of 15\% to 65\% by 2020. \citet{Hasson2009} showed that
the frequency of extreme bushfire events will increase to 1 or 2 per
year by the end of the 21\textsuperscript{th} century. \citet{Clarke2011}
argue that a higher bushfire risk potential is expected in the 21\textsuperscript{th }century.
This study also shows evidence that future bushfire seasons could
start earlier and be more prolonged. 

As shown in this review, most studies use the FFDI to describe fire
weather in Australia. Despite its acceptance by the scientific community,
the FFDI has significant shortcomings. First of all, empirical research
is the basis of this index. Therefore, its equation may not represent
the actual physical relationship between its variables. The index
does not consider the influence of topography on fire weather. Additionally,
the calibration of the FFDI considered the vegetation characteristics
of southeastern Australia. Therefore, it may not represent fire conditions
in other regions of the continent. Even the language of its fire danger
ratings have also been questioned \citep{Teague2010}. After the ``Black
Saturday'' fires, the FFDI ratings were updated to represent catastrophic
events like this. Moreover, \citet{Teague2010} suggest the development
of a new index. However, proposals for a new Australian fire weather
index have not been published in the scientific literature. 

The criticisms to the FFDI have been balanced with studies that show
its efficiency. For example, \citet{Dowdy2009} presented a comparative
study of the FFDI and the widely used FWI. The investigation mathematically
deconstructed the two indices to establish comparisons. The study
found that they offer comparable results in the Australian context.
Albeit its limitations, the FFDI continues to be the stepping stone
of fire weather research in Australia. 


\section{Fire weather forecasting }

The FFDI and other fire weather indices estimate bushfire risk based
on current\textemdash daily\textemdash weather conditions. However,
using these indices to forecast seasonal fire weather is a more complex
task. For example, using dynamical models requires extensive computing
resources \citep{Roads2005}. Other studies have explored the feasibility
of using ENSO indices to predict fire danger using statistical approaches
\citep{Kitzberger2002,Nicholls2007,Harris2013}. This review focuses
on this approach. This section reports on the influence of ENSO on
fire weather and bushfire events. It also discusses the use of other
climate modes of variability in fire forecasting. Finally, the review
also analyses some specific climate influences that foster fire activity
in South America. 

ENSO has a significant influence on fire occurrence in several regions
of the world \citep{Carmona-Moreno2005}. For example, \textquotedblleft La
Nina\textquotedblright{} events increase fire activity in the southern
regions of the United States \citep{Swetnam1990,Beckage2003}. The
effect of late stages of ``La Ni\~na'' is the same in northern Argentina.
In fact, \citet{Kitzberger2001} demonstrated that fire activity linked
to ENSO is synchronized in regions of North and South America. In
contrast, \textquotedblleft El Ni\~no\textquotedblright{} events increase
fire occurrence in regions such as in Indonesia \citep{Wooster2012},
tropical Mexico \citep{Roman-Cuesta2003} and the Brazilian Amazon
region \citep{Barlow2004}. Despite its global influence, ENSO's role
on fire activity remains unexplored in most of the world. Australia
is perhaps an exception. 

In Australia, most studies about inter-annual climate variability
and bushfires have focused on ENSO. \citet{Williams1999} determined
that \textquotedblleft El Ni\~no\textquotedblright{} events drive fire
weather in most of the central and eastern Australia. \citet{Verdon2004}
showed that positive phases of the Inter-decadal Pacific Oscillation
exacerbate this influence. \citet{Harris2008} argues that increased
rainfall linked to ENSO produces more vegetation that burns on the
dry season in northern Australia. However, there are fewer studies
on fire activity and other climate modes of variability. 

The influence of IOD and SAM on fire acitivity in Australia is less
understood. For example, \citet{Cai2009} found that positive phases
of the IOD produce hot and dry conditions during in spring in southeast
Australia. This situation leads to more available fuel loads and extreme
fire events in summer. Even though SAM is an important climate driver
for Australia, its relationship with fire activity has not been investigated.
Understanding the links between climate modes of varability and bushfires
is essential for forecasting purposes. 

The Australian Bureau of Meteorology includes ENSO's state in their
seasonal outlook reports (e.g. see http://www.bom.gov.au/climate/ahead/,
accessed 9 March 2016). These outlooks also include information about
temperature and precipitation. Fire management agencies use these
reports to qualitatively assess seasonal fire danger. However, some
authors suggest that ENSO indicators have potential to quantitatively
predict fire activity in Australia. For example, \citet{Nicholls2007}
argue that ENSO indices have this skill in Tasmania. \citet{Harris2013}
reached a similar conclusion for Victoria. These studies deserve a
closer analysis.

Tasmania and Victoria offer fire activity forecasting skill using
ENSO indices \citep{Nicholls2007,Harris2013}. \citet{Nicholls2007}
correlated area burnt data with climate variables and ENSO indicators.
Their analysis spanned the period 1952 to 2004. The climate variables
used in the correlation were rainfall and maximum temperature. The
ENSO predictors were the SOI and SST in the Coral Sea. The study found
that rainfall is the primary driver of fire activity in summer. This
research also established that SST values in winter are strongly correlated
with area burnt in summer. On the other hand, \citet{Harris2013}
performed a similar approach for Victoria. In this study, the authors
also correlated area burnt data with ENSO indices and climate variables.
Although, the period of analysis was shorter than Nichols et al. (2007).
The study spanned from 1972 to 2010. The climate variables used in
this research were vapor pressure at 0900 and 1500 hours, maximum
and minimum temperature, and rainfall. The ENSO indices used were
SOI, NINO1+2, NINO3, NINO4, and NINO3.4. The study determined that
vapor pressure at 1500 hours during the season September-October-November
offers the greatest forecasting skill. The study also found that ENSO
indices offer a moderate skill to forecast fire activity in Victoria.

The approach used in the investigations of \citet{Nicholls2007} and
\citet{Harris2013} have limitations. Using standard climate modes
of variability indices for regional seasonal forecasts could be misleading.
\citet{Verdon-Kidd2008a} argue that this approach fails to recognize
the specific regional influences of the climate modes of variability.
Additionally, their interaction in regions like Victoria is complex.
Therefore, investigating linear relationships is a short-coming. Moreover,
every climate event is different. Its trajectories vary widely depending
on the initial conditions of the system \citep{VonStorch2001}. This
characteristic is difficult to take into account in a statistical-empirical
forecasting model. Additionally, the long-term relationship between
climate modes of variability and weather changes over time. These
changes occur due to natural variability and anthropogenic climate
change \citep{Wang2013}. 

Despite its limitations, these studies provide useful information
and can be improved. \citet{Nicholls2007} and \citet{Harris2013}
rely on official fire activity records. However, it would be important
to contrast this information with alternative sources of fire data.
In fact, fire management decisions influence area burnt (e.g. fire
suppression, planned burns). Therefore, it would also be desirable
to use fire weather indices in the analyses. Using fire weather indices
could also be useful to extend the period of the correlations. Extending
these analyses would require computing fire weather indices with reanalysis
data (e.g. 20CR or ERA-20C). Finally, as acknowledged by \citet{Risbey2009b},
the influence of different climate modes of variability in Australia
varies depending on the region and season. Therefore, it would be
interesting to investigate the forecasting skill of other climate
modes of variability. In South America, the feasibility to forecast
extreme fire seasons has also been explored.

\citet{Chen2011} proposed a Fire Season Severity (FSS) index for
South America based on SST anomalies. The robust link between SST
and precipitation is the basis of the index \citep{Kousky1984,Ropelewski1987,Fernandes2011}.
The mechanism behind this link could be the displacement of the Intertropical
Convergence Zone (ITCZ) produced by variations in SST anomalies \citep{Zeng2008}.
A linear combination of the Oceanic Ni\~no Index (ONI) and the Atlantic
Multidecadal Oscillation Index (AMO) defines this index. Its design
used fire activity data from the Moderate Resolution Imaging Spectroradiometer
(MODIS) satellite \citep{Justice2002}. The calibration period was
2001-2009, testing against alternative satellite data for 2010. The
model was able to predict accurately extreme fire seasons with lead
times of 3 to 5 months. The authors emphasized the role of droughts
in producing severe fire seasons. However, the study did not quantify
the fire activity variability explained by precipitation. The short
period of analysis hampers the ability to assess this relationship.
Other forecasting schemes proposed in South America have had a local
focus.

\citet{Kitzberger2002} proposed a seasonal fire activity forecasting
scheme for northern Patagonia, Argentina. The logic behind the index
is simple. Extreme bushfires can occur in a switch from \textquotedblleft El
Ni\~no\textquotedblright{} to \textquotedblleft La Ni\~na\textquotedblright{}
phase \citep{Baisan1990,Swetnam1990,Kitzberger1997,Veblen1999}. In
the first phase, abundant precipitation fosters vegetation growth\textemdash and
therefore fuel availability\textemdash , followed by hot and dry conditions
that increase fire danger. The investigation used official fire data
from four national parks in Patagonia. The index uses 36 months of
ENSO indices data (SOI or NINO1+2) previous a fire season. This forecasting
tool can assess fire severity with three months in advance. This model
uses a \textquotedblleft high\textquotedblright{} versus \textquotedblleft low\textquotedblright{}
seasonal fire danger rating. The threshold for this rating is 250
Ha of burnt area (in the test with best results). The index was able
to explain 53\% of the variance and misclassified only 4 out of 43
fire seasons. However, the author acknowledges several limitations
with his model. Since every ENSO event is different, its arrival timing
might affect the results. Similarly, ENSO's teleconnection patterns
differ in each event \citep{Villalba1994}. The model also assumes
that fuel moisture as the only limiting factor. Although in reality,
ignition sources also play a role. Finally, it also assumes that the
relationship between ENSO and regional climate is stable over time.
However, as already noted in this review, this is not the case. 

There are no other seasonal fire danger forecasting schemes proposed
for South America. Although, the relationship between SST and fire
weather could be helpful to understand seasonal predictability. The
SST variability drives fire weather conditions in the Amazon region
and northern Patagonia \citep{Kousky1984,Ropelewski1987,Fernandes2011,Kitzberger2001,Kitzberger2002}.
On the other hand, strong \textquotedblleft El Ni\~no\textquotedblright{}
events bring heavy precipitation to the coasts of Ecuador and Peru
\citep{Rasmusson1982,Horel1986,Rodbell1999}. However, less is know
about the influence of SST on the Andean climate. 

\citet{Garreaud2009} reviewed the climate scientific literature of
the Andean region. In this review, the author generalizes about ENSO's
influence over the Andes. \textquotedblleft El Ni\~no\textquotedblright{}
events bring less than normal precipitation on the tropical Andes.
Conversely, it produces higher than normal precipitation in the sub-tropics.
In both latitudinal ranges, it produces warmer than average air temperature.
Overall, the opposite pattern prevails for temperature and precipitation
during \textquotedblleft La Ni\~na\textquotedblright{} years. The ENSO
signal on the tropical Andes contrasts with the torrential rains that
strong \textquotedblleft El Ni\~no\textquotedblright{} brings to the
coast. \citet{Garreaud2009} argues that this phenomenon occurs due
to a distortion of the Hadley cell circulation. 

\citet{Vuille2000} and \citet{Francou2004} studied the influence
of SST in the tropical Andes region. Both agree that \textquotedblleft El
Ni\~no\textquotedblright{} brings less precipitation and higher air
temperature. \citet{Francou2004} demonstrated that this effect produces
negative mass balances on the tropical Andean mountain glaciers. However,
\citet{Vuille2000} highlight that SST anomalies from the Pacific
Ocean only influence climate on the northwestern side of the Ecuadorian
Andes. On the other hand, SST anomalies from the tropical Atlantic
Ocean explain most of the climate variability in the western \textquotedblleft Cordillera\textquotedblright{}
.


\section{Research gaps}

Several research gaps emerge from this literature review. Overall,
there are weather and climate studies for South America. However,
the information is dispersed. Additionally, South America lacks information
about fire activity and its relationship with weather. The Andean
region is perhaps one of the areas with less information. On the other
hand, the absence of fire weather studies before the nineteen-seventies
seems to be a common factor in Australia. This gap is further discussed.

The three research areas discussed in this review lack extended periods
of analyses. These gaps occur due to the unavailability of fire activity
records and weather station data. However, the compilation and comparison
of existing fire records could be useful in extending these analyses.
Additionally, the use of reanalysis products might also be an alternative
to address this gap.

In addition to extending the period of analyses, the investigation
of the differences between bushfire and heatwave weather patterns
is an interesting research endeavour. The literature considers them
related events, although \citet{Reeder2015} argue they are different.
However, there is no comprehensive study of how they differ. This
investigation is important because it could improve weather forecasting
capabilities.

The development and improvement of fire weather indices is another
area that offers research opportunities. In Australia, the McArthur
Forest Fire Danger Index would benefit from a revision. In fact, as
suggested by \citet{Teague2010}, perhaps a new index would be required. 

On the other hand, several studies have evaluated seasonal forecasting
potential using ENSO indices \citep{Nicholls2007,Harris2013}. However,
these studies did not propose a model or simple index. Therefore,
it would be interesting to test a new fire weather index that offers
seasonal predictability skill in Australia. 


\section{Summary}

The literature review has discussed four main areas. First of all,
it compared and contrasted bushfire and heatwave weather patterns
literature. Secondly, it discussed bushfire danger studies based on
the McArthur Forest Fire Danger Index. The review also examined studies
that aim to forecast seasonal fire danger using ENSO indicators. Finally,
the research gaps have been synthesised.

The next chapter addresses two of the identified gaps. It aims to
describe the differences between bushfire and heatwave weather patterns.
It also extends the period of analysis of these two types of events.
This investigation was conducted for Victoria, Australia.
